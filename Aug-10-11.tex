\Lecture{Jayalal Sarma}{August, 10 2015}{05}{Orbit-Stabilizer Lemma}{Ameya Panse}{$\delta$}

We first start with some notations and definitions. Order of a group $G$ is number of elements in the group, that is $|G|$. Let $H \le G$ and $g \in G$. The right coset of $H$ in $G$ is defined as 
$Hg = \{hg \mid h \in H\}$. If the mutliplication is on the left, we call it the left coset. That is, $gH = \{gh \mid h \in H\}$. In general, it is not necessary that the left and right cosets are the same.
If $H$ is a group such that left and right cosets are the same element-wise, (that is, $\forall g \in G, Hg= gH$), $H$ is a said to be a normal subgroup of $G$.

Let $H$ be a normal subgroup of $G$. Choose one element from each of these as a representative of the set (say $[g]$ denote the coset representative of the coset $Hg = gH$). These elements have a group structure among them. This requires a proof, which we will come to in the next few lectures.

\section{Group Action and Orbits}

Let $G$ be a Subgroup of $S_n$. Let $\alpha \in [n]$ and $g \in G$. We denote by $\alpha^g$ is the image of $\alpha$ under the permutation $g$. 
Orbit of an element $\alpha$ in $G$ is the set of elements it gets mapped to under permutations in $G$. More formally,

\begin{definition}[\textbf{Orbit of $\alpha$ in $G$}]
The orbit of $\alpha$ in $G$ is defined as
\begin{center}
$\alpha^G = \{\alpha^g \mid g \in G\}$
\end{center}
\end{definition}

This defines a natural relation among elements of $[n]$.
$$\alpha \backsim \beta \leftrightarrow \exists g \in G, \alpha^g = \beta$$

We check that this is an equivalence relation. $e \in G$, where e is the identity element. Hence, $\alpha \backsim \alpha$. Let $\alpha \backsim \beta$. Thus $\exists g \in G, \alpha^g = \beta$. Hence, $\alpha = \beta^{g^{-1}}$. Thus, $\beta \backsim \alpha$. The relation $\backsim$ is an equivalence relation. For transitivity, let $\alpha \backsim \beta, \beta \backsim \gamma$. By definition, $\exists g_1, g_2 \alpha^{g_1} = \beta, \beta^{g_2} = \gamma$. By composition of permutations, $(\alpha^{g_1})^{g_2} = \gamma$. Hence, $\alpha \backsim \gamma$.

The stabilizer of $\alpha$ in $G$ is $$G_{\alpha} = \{g \mid \alpha^g = \alpha\}$$. This is the set of elements in $G$ which sends $\alpha$ to $\alpha$ itself. 

\subsection{Orbit-Stabilizer Lemma}

Is there any connection between the number of permutations in $G$ that fixes $\alpha$ and the number of elements to which $\alpha$ can be taken to? The orbit stabilizer lemma, which is an easy consequences of Lagrange's theorem gives a neat answer.

\begin{lemma}[\textbf{Orbit-Stabilizer Lemma}]
Let $G\leq S_n$. Then for any $\alpha \in [n]$,
$$|\alpha^G|*|G_{\alpha}| = |G|$$
\end{lemma}
\begin{proof}
We quickly observe that $G_{\alpha}$ is a Subgroup of $G$. Indeed, identity belongs to $G_\alpha$ trivially. If $g$ and $g'$ both fixes $\alpha$, then so does $gg'$ and $g'g$. If $g$ fixes $\alpha$, then so does $g^{-1}$. Since $G_{\alpha}$ forms a Subgroup of $G$, by Lagrange's Theorem,
$$\frac{|G|}{|G_{\alpha}|} = \mbox{number of distinct right cosets of $G_\alpha$ in $G$}$$ 


To complete the proof, it suffices to argue that the number if distinct cosets of $G_\alpha$ in $G$ is the size of the orbit of $\alpha$ under the action of $G$. We now show the bijection.

Let $\beta \in \alpha^G$ via $h \in G$. That is,$\alpha^h = \beta$. Consider the map,
$$\Gamma : \beta \rightarrow \{g \in G| \alpha^g = \beta \} $$

We first show that this is well-defined map between the elements in the orbit of $\alpha$ to the cosets. For that, we first show that  $\{g \in G| \alpha^g = \beta \}$ is indeed a right coset of $G_\alpha$ in $G$. To argue this, it suffices to show that 
\begin{align*}
\Gamma(\beta) = \{g \in G\mid \beta = \alpha^g\} &= \{ g \in G \mid \alpha^h = \alpha^g \} = \{ g \in G \mid \alpha^{gh^{-1}} = \alpha\} \\
 & = \{ g \in G \mid gh^{-1} \in G_{\alpha} \} = \{ g \in G \mid g \in G_{\alpha}h = G_\alpha h
\end{align*}
Thus $\Gamma$ is a function.

We now show that the function $\Gamma$ is injective. Let $\beta$ and $\gamma$ be two different elements of the orbit of $\alpha$ via the group elements $h_{\beta}$ and $h_{\gamma}$. That is, 
$$ \Gamma(\beta) = \{ g \in G \mid \alpha^g = \beta \} $$
$$ \Gamma(\gamma) = \{ g \in G \mid \alpha^g = \gamma \} $$
Indeed, these two sets cannot have an intersection since $\beta \neq \gamma$. Thus, $\Gamma(\beta) \ne \Gamma(\gamma)$.

We now argue that the function $\Gamma$ is surjective. Consider any coset $C = G_\alpha g$ of $G_\alpha$ in $G$, where $g \in G$. We show that there is a $\beta \in \alpha^G$ such that $\Gamma(\beta) = C$. Indeed, define $\beta$ to be $\alpha^g$. Consider any $h \in G$ such that $\alpha^h = \beta$. 
\jsay{Surjectivity proof to be completed}
\end{proof}

\section{Graph Automorphism and Graph Isomorphism}
We define the problems that we are interested in, technically.

\begin{problem}[\textsc{Graph Isomorphism Problem (GI)}]
Given a graph $X_1=(V_1,E_1)$ and $X_2=(V_2,E_2)$, decide if $X_1\cong X_2$ or not.
\end{problem}
\begin{problem}[\textsc{Graph Automorphism Problem (GA)}]
Given a graph $X=(V,E)$, compute a Generating Set for $Aut(X)$.
\end{problem}

\begin{problem}[\textsc{Graph Rigidity Problem (GR)}]
Given a graph $X=(V,E)$, decide if $Aut(X)$ is trivial or not.
\end{problem}

\begin{problem}[\textsc{Counting Isomorphisms (\#GI)}]
Given a graph $X_1=(V_1,E_1),X_2=(V_2,E_2)$, output the number of Isomorphisms from $X_1$ to $X_2$, in binary.
\end{problem}

\begin{problem}[\textsc{Counting Automorphisms (\#GA)}]
Given a graph $X=(V,E)$, compute the size of the automorphism group, $|Aut(X)|$, in binary.
\end{problem}

\begin{problem}[\textsc{Computing the Isomorphism (ISO)}]
Given a graph $X_1=(V_1,E_1),X_2=(V_2,E_2)$, output an isomorphism map between $V_1$ and $V_2$ if it exists.
\end{problem}

We use the notion of polynomial time reducibility between two problems. Given two problems $A,B$, we say that $A \le B$ (A reduces to B), if given a polynomial time (in terms of input size $n$) algorithm for $B$, we can give out a polynomial time algorithm for $A$. We quickly observe some easy relationship among these problems. Indeed, if we can solve $GA$, we can solve $GR$ as well. Given a graph $X$, to check if it is rigid, it is only a matter of checking if there is a nontrivial element in the generating set for the group $Aut(X)$. Hence we conclude that 
$GR \le GA$. The same is the case for $GR$ and $\#GA$ as well. That is, $GR \le \#GA$. Similarly, if we can compute the isomorphism, we can decide it as well. Trivially, $GI \le ISO$. It is interesting to think about whether $\#GA$ can be done using $\GA$. That is, given the generating set of a permutation group (that is, subgroup of $S_n$), can we compute the size of the generated group?

\Lecture{Jayalal Sarma}{August, 11 2015}{06}{A Closer Look at Graph
Isomorphism and Automorphism}{Ameya Panse}{$\gamma$}

\section{Another related Problem}
Let $X=(V,E)$ be a graph. Consider the vertex set 
$V$ to be divided into $c$ color classes. That is there exists a function $\Psi : V(X) \rightarrow [c]$,
where $i \in [c]$ denotes a color and the $i^{th}$ color class is $\Psi^{-1}(i)$.

We say a graph $X=(V,E)$ is {\em $C$-colored} if
the vertex set $V$ is partitioned into $C$ color classes.

Consider the colored version of GI,

\textbf{Colored Graph Isomorphism [CGI]:} Given two $C$-colored Graphs $(X_1,\Psi_1)$ and $(X_2,\Psi_2)$
decide if there exists $\sigma : V(X_1) \rightarrow V(X_2)$ such that for all
$(u,v) \in V(X_1)\times V(X_2)$,  $(u,v) \in E(X_1)$ if and only if $(\sigma(u), \sigma(v)) \in E(X_2) $
 and for every $u \in V(X_1),\Psi_1(u) = \Psi_2(\sigma(u))$ .

\section{Relationships Among the Problems}

\subsection{GI $\le$ CGI}
GI trivially reduces to CGI.	Set $c=1$, and color all the vertices with the same color.


\subsection{CGI $\le$ GI}
Given $(X_1,\Psi_1)$ and $(X_2,\Psi_2)$ construct graphs $X_1'$ and $X_2'$ such that 
\[ ((X_1,\Psi_1),(X_2,\Psi_2))\in CGI \iff (X_1',X_2')\in GI \]
The construction of $X_1'$ and $X_2'$ are as follows :
For every vertex $u \in V(X_1)$ such that $u \in \Psi_1^{-1}(i)$:
\begin{enumerate}
\item Add $ni$ extra vertices to $X_1$.
\item Add edges from each of the extra vertices to $u$.
\end{enumerate}
Do the same for $(X_2,\Psi_2)$ to get $X_2'$.\\

\begin{claim}
$((X_1,\Psi_1),(X_2,\Psi_2))\in CGI \iff (X_1',X_2')\in GI$
\end{claim}
\begin{proof}
($\Rightarrow$)Let $(X_1,\Psi_1)$ and $(X_2,\Psi_2)$ $\in CGI$. To show that $X_1' \cong X_2'$. Since  $(X_1,\Psi_1)$ and $(X_2,\Psi_2)$ $\in CGI$ we know that there exists $\sigma : V(X_1) \rightarrow V(X_2)$ such that $\forall (u,v) \in V(X_1)\times V(X_2), (u,v) \in E(X_1)$ if and only if $(\sigma(u), \sigma(v)) \in E(X_2) $ and $\forall u \in V(X_1) , \Psi_1(u) = \Psi_2(\sigma(u))$. Additionally map the extra vertices added correspondingly. Thus $X_1' \cong X_2'$.\\
($\Leftarrow$)Let $X_1' \cong X_2'$. To show that $(X_1,\Psi_1) \cong (X_2,\Psi_2)$.\\
$X_1' \cong X_2'$, hence $ \exists \sigma : V(X_1') \rightarrow V(X_2'), \forall (u,v) \in E(X_1'), (\sigma(u),\sigma(v)) \in E(X_2')$.\\
If possible let there exist $ u \in v(X_1), u \in \Psi^{-1}(i)$ such that $\sigma(u) \notin V(X_2)$. That is, u is mapped to one of the extra vertices. But $u \in X_1'$ and $deg(u) \ge ni$, whereas degree of any extra vertex is 1. Hence, we have a contradiction.

Now, if possible let $u \notin \Psi_2^{-1}(i)$. Hence $\sigma(u) \in \Psi_2^{-1}(j)$ and $j \neq i$. Note that $ni + n$ \textgreater $deg(u) \ge ni \implies n(i+1)$ \textgreater $deg(u) \ge ni$.

And, $\sigma(u) \in \Psi_2^{-1}(j) \implies n(j+1)$ \textgreater $deg(u) \ge nj$. Since both of these can not be true simultaneously, we have a contradiction.

Hence, $\sigma(u) \in V(X_2) , \sigma(u) \in \Psi_2^{-1}(i)$. Thus, $(X_1,\Psi_1) \cong (X_2,\Psi_2)$.

We have made only one query to GI. Hence, the reduction is polytime.
\end{proof}



\subsection{Computing Isomorphism $\le GI$}
We show that computing isomorphism reduces to colored graph isomorphism. As $CGI\leq GI$ we have $Computing Isomorphism \le GI$.  Given $X_1,X_2$ Output a permutation that maps $X_1$ to $x_2$.
The Reduction is as follows:
\begin{enumerate}
\item Check if $X_1 \cong X_2$. If NO, end.
\item For each vertex $i \in V_1$ 
	Color $i$ with color $C_i$.
	\begin{itemize}
	\item Color $j \in V_2$ that is not already colored with $C_i$, temporarily.
	\item Query CGI to check if $X_1\cong X_2$.
	\item Repeat on $j$ untill you get a "yes" answer. Fix color of $j$ as $C_i$.
	\end{itemize}
Ouput the permutation.
\end{enumerate}
Here each vertex is colored with a different color, and hence we have a permutation.\\
Also, if the graphs are isomorphic, then there will exists an $j \in V_2$, such that we get a yes answer.

Observe that we make atmost $O(n^2)$ queries to $CGI$ which in turn makes a single query to $GI$. Thus the reduction is polytime.


\subsection{GI $\le$ GA}
Take $X = X_1 \cup X_2$. We need to construct a graph $G$ such that the generating set $S$ of $Aut(G)$ enables to decide if $X_1\equiv X_2$.

\begin{claim}
$X_1 \cong X_2$ if and only if there exists a $\sigma \in S$ such that $\sigma$ maps atleast one vertex in $X_1$ to a vertex in $X_2$.
\end{claim}

\begin{proof}
\textbf{Case 1 : $X_1$ and $X_2$ are connected.}

($\Rightarrow$) Suppose $X_1 \cong X_2$. Then there exists a $\tau$ which is an isomorphism from $X_1,$ to $X_2$. $\tau \in Aut(X)$. Hence, there is a $\sigma$ that maps a vertex in $X_1$ to a vertex in $X_2$.\\

($\Leftarrow$) Let there exist a $\sigma\in S$ such that $\sigma$ maps $u \in X_1$ to a vertex in $X_2$. Let $v \in X_1$ be such that $\sigma(u) \in X_1$. Since $X_1$ is connected $u,v$ are connected. But $\sigma(u) \in X_2, \sigma(v) \in X_1$ are not connected which is a contradiction. Therefore for $\sigma\in S$, $\sigma$ maps all the vertices in $X_1$ to $X_2$. Thus $X_1 \cong X_2$.

\textbf{Case 2 : $X_1$ and $X_2$ are not connected.}

In case the two are not connected, add an extra vertex to both the graphs that is adjacent to all the vertices in the corresponding graphs. Since, the new vertices have degree $n$, they can only be mapped to each other.
\end{proof}

Thus GI $\le$ GA.



