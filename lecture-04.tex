\Lecture{Jayalal Sarma M.N.}{Aug 7, 2013}{4}{Graphs, Groups and Generators}
\noindent
Preamble


\section{Graph Isomorphism}


\begin{defn}(Graph Isomorphism.)
Let $G_1=(V_1,E_1)$ and $G_2=(V_2,E_2)$ be graphs. We say $G_1\stackrel{\sim}{=}G_2$(read as $G_1$ is \em isomorphic to $G_2$) if there exists a bijection $\sigma : V_1\rightarrow V_2$ such that $\forall (u,v) \in V_1\times V_2$ we have
\begin{center}
$(u,v)\in E_1 \iff (\sigma(u),\sigma(v))\in E_2$
\end{center}
\end{defn}

In other words, we say a graph $G_1$ is isomorphic to $G_2$ if there exists a relabeling of the vertices in $G_1$  such that the the adjacency and non-adjacency relationships in $G_2$ is preserved. 
\begin{obs}
If $|V_1|\neq |V_2|$ we have that $G_1$ is not isomorphic to $G_2$.
\end{obs}

The graph isomorphism problem is stated as follows. 
\begin{center}
\fbox{
\begin{minipage}{10 cm}\textbf{PROBLEM : GRAPH ISOMORPHISM}\\
\textbf{Input} : $G_1=(V_1,E_1), G_2=(V_2,E_2)$\\
\textbf{Output} : Decide if $G_1\stackrel{\sim}{=}G_2$ or not.
\end{minipage}
}
\end{center}

A natural question to ask in this setting is that if there is an isomorphism from a graph $G$ to itself. 

Let $[n]=\{1,2,\ldots,n\}$. Let $S_n$ denote the set of all permuatations from the set $[n]$ to $[n]$.

\begin{defn}(Graph Automorphism.) 
Let $G=(V,E)$ be a graph. An automorphism of $G$ is a bijection $\sigma:V\rightarrow V$ such that $\sigma(G)= G$. Let
\begin{center}
$Aut(G)= \{\sigma | \sigma\in S_n \text{~and~} \sigma(G)= G \}$
\end{center}
be the set of all automorphisms of $G$. 
\end{defn}



The graph automorphism problem is stated as follows. 
\begin{center}
\fbox{
\begin{minipage}{10 cm}\textbf{PROBLEM : GRAPH AUTOMORPHISM}\\
\textbf{Input} : A graph $G=(V,E)$\\
\textbf{Output} : Construct $Aut(G)$.
\end{minipage}
}
\end{center}



\begin{obs}
Let $\tau:[n]\rightarrow[n]$ be the identity permuataion. That is, for all $i\in[n],\tau(i)=i$. Then by definition $\tau\in Aut(G)$ for any graph $G=(V,E)$. This identity permuataion $\tau$ is in $Aut(G)$. 
\end{obs}


Are there other permutations from $[n]$ to $[n]$ that are in the set $Aut(G)$ ? How large can the set $Aut(G)$ be ?

Formally, the graph rigidity problem is stated as follows.
\begin{center}
\fbox{
\begin{minipage}{10 cm}\textbf{PROBLEM : GRAPH RIGIDITY}\\
\textbf{Input} : A graph $G=(V,E)$\\
\textbf{Output} : Decide if $Aut(G)$ is trivial or not.
\end{minipage}
}
\end{center}

\begin{obs}
Let $G$ be the complete graph on $n$ vertices. Then $|Aut(G)|=n!$.  
\end{obs}

Is the set $Aut(G)$ just a set or does it have more algebraic structure ? 

\begin{defn}(Groups.)
A set $G$ together with a binary operation $*$ is said to be a group if the following four consitions are met
\begin{itemize}
\item \textbf{Closure} : $a,b\in G$, the element $a*b\in G$.
\item \textbf{Associative} : For any $a,b,c\in G$, we have $(a*b)*c = a*(b*c)$.
\item \textbf{Existence of Identity} : For any $a\in G$ there exists a unique element $e\in G$ such that $a*e =e*a =a$. 
\item \textbf{Existence of Identity} : For any $a\in G$ there exists a unique element $b\in G$(denoted by $a^{-1}$) such that $a*b =b*a =e$. 
\end{itemize}
\end{defn}

\begin{eg}
\begin{itemize}
\item $S_n$ forms a group under composition.
\item $(\mathbb{Z}_5,+)$ is a group.
\end{itemize}
\end{eg}

From now on, we will use the letter $X$ to denote a graph and $G$ to denote a group.

\begin{notn} 
Let $(G,*)$ be a group. Let $H\subseteq G$ such that $(H,*)$ also forms a group. We say $H$ is a {\em subgroup} of $G$ and denote by $H\leq G$.   
\end{notn}


\begin{ex}
For any graph X, the set $Aut(X)$ forms a group under the composition operation. That is, $Aut(G)\leq S_n$.
\end{ex}


Let $(G,+)$ be a finite group and $g\in G$ be an element. Let $g^2 =  g*g ,g^3 =  g*g*g $. Similarly $g^k = \underbrace{g*g*\cdots * g}_\text{k times}$. Now consider the set $H=\{g,g^2,g^3,\ldots\}$. Since $(G,*)$ is a finite group there must exist a $k$ such that $g^k=g$.

\begin{lemma}
Let $(G,+)$ be a finite group and $g\in G$ be an element. Let $H=\{g,g^2,g^3,\ldots\}$ be a set of elements. The unique identity $e$ in $G$ is in $H$.
\end{lemma}
\begin{proof}
By definition, $g^k = g^{k-1}*g = g$. \\
As $(G,*)$ is a group, we have have the inverse of $g^{-1}$ in $G$.\\
Therefore
\begin{center}
$g^{k-1}*g*g^{-1} = g*g^{-1} = e$ 
\end{center}
\end{proof}


