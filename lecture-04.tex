\Lecture{Jayalal Sarma M.N.}{Aug 7, 2015}{4}{Graphs, Groups and Generators}{Ramya C}

In this lecture we will pose three graph theoretic questions and find answers using approaches in algebra.

\section{Graph Isomorphism, Automorphism and Rigidity}

\begin{definition}(Graph Isomorphism.)
Let $G_1=(V_1,E_1)$ and $G_2=(V_2,E_2)$ be graphs. We say $G_1\stackrel{\sim}{=}G_2$ (read as $G_1$ is \em isomorphic to $G_2$) if there exists a bijection $\sigma : V_1\rightarrow V_2$ such that $\forall (u,v) \in V_1\times V_2$ we have

\begin{center}
$(u,v)\in E_1 \iff (\sigma(u),\sigma(v))\in E_2$
\end{center}
\end{definition}

In other words, we say a graph $G_1$ is isomorphic to $G_2$ if there exists a relabeling of the vertices in $G_1$  such that the the adjacency and non-adjacency relationships in $G_2$ is preserved. 
\begin{observation}
If $|V_1|\neq |V_2|$ ,then $G_1$ is not isomorphic to $G_2$.

\end{observation}

The graph isomorphism problem is stated as follows. 
\begin{center}
\fbox{
\begin{minipage}{10 cm}\textbf{PROBLEM : GRAPH ISOMORPHISM}\\
\textbf{Input} : $G_1=(V_1,E_1), G_2=(V_2,E_2)$\\
\textbf{Output} : Decide if $G_1\stackrel{\sim}{=}G_2$ or not.
\end{minipage}
}
\end{center}

A natural question to ask in this setting is that if there is an isomorphism from a graph $G$ to itself. 

Let $[n]=\{1,2,\ldots,n\}$. Let $S_n$ denote the set of all permuatations from the set $[n]$ to $[n]$.

\begin{definition}(Graph Automorphism.) 
Let $G=(V,E)$ be a graph. An automorphism of $G$ is a bijection $\sigma:V\rightarrow V$ such that $\sigma(G)= G$. Let
\begin{center}
$Aut(G)= \{\sigma | \sigma\in S_n \text{~and~} \sigma(G)= G \}$
\end{center}
be the set of all automorphisms of $G$. 
\end{definition}



The graph automorphism problem is stated as follows. 
\begin{center}
\fbox{
\begin{minipage}{10 cm}\textbf{PROBLEM : GRAPH AUTOMORPHISM}\\
\textbf{Input} : A graph $G=(V,E)$\\
\textbf{Output} : Construct $Aut(G)$.
\end{minipage}
}
\end{center}



\begin{observation}
Let $\tau:[n]\rightarrow[n]$ be the identity permuataion. That is, for all $i\in[n],\tau(i)=i$. Then by definition $\tau\in Aut(G)$ for any graph $G=(V,E)$. This identity permuataion $\tau$ is in $Aut(G)$. 
\end{observation}


Are there other permutations from $[n]$ to $[n]$ that are in the set $Aut(G)$ ? How large can the set $Aut(G)$ be ?

Formally, the graph rigidity problem is stated as follows.
\begin{center}
\fbox{
\begin{minipage}{10 cm}\textbf{PROBLEM : GRAPH RIGIDITY}\\
\textbf{Input} : A graph $G=(V,E)$\\
\textbf{Output} : Decide if $Aut(G)$ is trivial or not.
\end{minipage}
}
\end{center}

\begin{observation}
Let $G$ be the complete graph on $n$ vertices. Then $|Aut(G)|=n!$.  
\end{observation}

Is the set $Aut(G)$ just a set or does it have more algebraic structure ? 


\section{Groups and Generators}

\begin{definition}(Groups.)
A set $G$ together with a binary operation $*$ is said to be a group if the following four consitions are met
\begin{itemize}
\item \textbf{Closure} : $a,b\in G$, the element $a*b\in G$.
\item \textbf{Associative} : For any $a,b,c\in G$, we have $(a*b)*c = a*(b*c)$.
<<<<<<< HEAD
\item \textbf{Existence of Identity} : For any $a\in G$ there exists a unique element $e\in G$ such that $a*e =e*a =a$. 
\item \textbf{Existence of Inverse} : For any $a\in G$ there exists a unique element $b\in G$(denoted by $a^{-1}$) such that $a*b =b*a =e$. 

\end{itemize}
\end{definition}

\begin{example}
\begin{itemize}
\item $S_n$ forms a group under composition.
\item $(\mathbb{Z}_5,+)$ is a group.
\end{itemize}
\end{example}

From now on, we will use the letter $X$ to denote a graph and $G$ to denote a group.

\begin{remark} 
Let $(G,*)$ be a group. Let $H\subseteq G$ such that $(H,*)$ also forms a group. We say $H$ is a {\em subgroup} of $G$ and denote by $H\leq G$.   
\end{remark}


\begin{exercise}
For any graph X, the set $Aut(X)$ forms a group under the composition operation. That is, $Aut(G)\leq M$ where $M=(S_n,\circ)$ is the permutation group.
\end{exercise}

Let $(G,+)$ be a finite group and $g\in G$ be an element. Let $g^2 =  g*g ,g^3 =  g*g*g $. Similarly $g^k = \underbrace{g*g*\cdots * g}_\text{k times}$. Now consider the set $H=  \{g,g^2,g^3,\ldots\}$. Since $(G,*)$ is a finite group there must exist a $k$ such that $g^k=g$ in $H$.

\begin{lemma}
Let $(G,+)$ be a finite group and $g\in G$ be an element. Let $H=\{g,g^2,g^3,\ldots\}$ be a set of elements. The unique identity $e$ of $G$ is in $H$.
\end{lemma}
\begin{proof}
Since $(G,*)$ is a finite group there must exist a $k$ such that $g^k=g$ in $H$. By definition, $g^k = g^{k-1}*g = g$. As $(G,*)$ is a group, $g^{-1}$ exists in $G$.\\
Therefore,
\begin{center}
$g^{k-1}*g*g^{-1} = g*g^{-1} = e$ 
\end{center}
\end{proof}

\begin{definition}[Generator]
Let $(G,+)$ be a finite group and $g\in G$ be an element. We say an element $g\in G$ is a generator of the set $H$ if for every element $h\in H$ there exists a $m$ such that $h=g^m$.  (denoted by $H=<g>$).
\end{definition}

\begin{observation}$H = <g>$ is a subgroup of $G$. That is, $H\leq G$.
\end{observation}

\begin{example}
\begin{itemize}
\item $<1> = (\mathbb{Z}_5,+)$
\end{itemize}
\end{example}



\begin{definition}An group $(G,*)$ that can be generated by a single element is called a {\em cyclic group}. For instance, $(\mathbb{Z}_5,+)$.
\end{definition}

Not every group is cyclic. For instance $S_3$ is not cyclic.

<<<<<<< HEAD
\begin{definition}(Generating set.)
Let $S\subseteq G$ be the set $\{u_1,\ldots,u_k\}$. $S$ is said to be {\em generating} if $<S>=G$.  
\end{definition}

Having observed that $Aut(X)$ could have potentially be of exponenetial size, it is natural to look for generating sets of small size.

Given a graph $X=(V,E)$ where $|V|=n$, does $Aut(X)$ have a generating set of size $\poly(n)$?

\subsection{Lagrange's theorem}

Let $(G,*)$ be a group. Let $H\leq G$. For any $g\in G$, define the right coset of $H$ in $G$ to be
\begin{center}
$Hg= \{hg \mid h\in H \}$
\end{center}
 

Let $g_1,g_2\in G$ and $Hg_1,Hg_2$ be the corresponding right cosets.  Are there elements that belong to more than one coset of $H$ in $G$? Is $Hg_1\cap Hg_2 \neq \phi$ ? If yes, then the set of right cosets of $H$ in $G$ form a partition of the ground set of $G$. In that case, how many such cosets are required to cover the entire set $G$ ? Let us answer these two questions.

\begin{lemma}
\label{lag1}
Let $g_1,g_2\in G$ and $Hg_1 = \{hg_1\mid h\in H\},Hg_2 = \{hg_2\mid h\in H\}$. Then 
\begin{center}
$Hg_1=Hg_2$ or $Hg_1\cap Hg_2 = \phi$.
\end{center}
\end{lemma}
\begin{proof}
If $g_1=g_2$, then by definition $Hg_1=Hg_2$. Therefore let $g_1\neq g_2$. We will prove : If $Hg_1\cap Hg_2 \neq \phi$ then $Hg_1 = Hg_2$. Let $Hg_1\cap Hg_2 \neq \phi, g\in Hg_1\cap Hg_2$ we will show 
\begin{itemize}
\item[(i)] $Hg_1\subseteq Hg_2$ ; and
\item[(ii)] $Hg_2\subseteq Hg_1$.
\end{itemize}

Since $g\in Hg_1$ we know that there exists a $h_1\in H$ such that $g=h_1g_1$. Similarly $g\in Hg_2$ suggests that there exists a $h_2\in H$ such that $g=h_2g_2$.
\begin{center}
$h_1g_1=h_2g_2=g$
\end{center}   
As $(H,*)$ is a group by itself, $h_1^{-1}$ and $h_2^{-1}$ exists.
\begin{equation}
\label{eqn1}
g_1= h_1^{-1}h_2g_2
\end{equation}

\begin{equation}
\label{eqn2}
g_2 = h_2^{-1}h_1g_1
\end{equation}

\begin{itemize}
\item[(i)] $Hg_1\subseteq Hg_2$ \\
Let $g'\in Hg_1$. This implies there exists a $h'\in H$ such that $g'=h'g_1$. Therefore,
\begin{align*}
g'&= h'g_1\\
g'&= h'(h_1^{-1}h_2g_2) \hspace*{10mm} [\text{By equation \eqref{eqn1}}]
\end{align*} 
By closure property in $(H,*)$, we have $h''=h'h_1^{-1}h_2\in H$. Therefore $g'=h''g_2$, $g'\in Hg_2$.
\item[(ii)] $Hg_2\subseteq Hg_1$ \\
Let $g'\in Hg_2$. This implies there exists a $h'\in H$ such that $g'=h'g_2$. Therefore,
\begin{align*}
g'&= h'g_2\\
g'&= h'(h_2^{-1}h_1g_1) \hspace*{10mm} [\text{By equation \eqref{eqn2}}]
\end{align*} 
By closure property in $(H,*)$, we have $h''=h'h_2^{-1}h_1\in H$. Therefore $g'=h''g_1$, $g'\in Hg_1$.
\end{itemize}
\end{proof}

\begin{lemma}
\label{lag2}
For every $g\in G$, $|Hg|=|H|$.
\end{lemma}
\begin{proof}
By construction, for every element in $H$ there exists an element in $Hg$. So $|Hg|\leq |H|$. 

Let us show : $|H|\leq |Hg|$.Suppose not, $|Hg|<|H|$. Then there exists $h_1,h_2\in H, h_1\neq h_2$ such that $h_1g = h_2g$. Since $(G,*)$ is a group, $g^{-1}$ exists. We have $h_1gg^{-1} = h_2gg^{-1}$ which implies $h_1=h_2$, a contradiction.
\end{proof}

\begin{theorem}(Lagrange's theorem.)
\label{lag-thm}
Let $(G,*)$ be a group and $H\leq G$. Then $|H|$ divides $|G|$.
\end{theorem}
\begin{proof}
Direct consequence of Lemmas \ref{lag1} and \ref{lag2}
\end{proof}


\begin{observation}
\label{obs-genset}
Let $H=<g>$ and $H'=<H,g'>$ where $g'\in G\backslash H$. Then $H\leq H'\leq G$. We have $g\in H'\backslash H$, therefore $|H'|>|H|$. $H'\leq H$. By Theorem \ref{lag-thm}, $|H'|\geq 2|H|$. This shows that every group has a generating set of size $\log|G|$.
\end{observation}


\begin{remark}
We know that $Aut(X)\leq S_n$ for any graph $X=(V,E)$. By Observation \ref{obs-genset} $Aut(X)$ has a generating set of size $\log|S_n| = log(n!) \in \mathcal{O}(n\log n)$ by Stirling's approximation.
\end{remark}  

