\Lecture{Jayalal Sarma}{August 19, 2015}{11}{Algorithm for Coset Representatives, Pointwise Stabilizer, Membership Testing}{Samir Otiv}{$\alpha$}

\section{The Pointwise Stabiliser Problem}
Input: Generators of $G$ and a set of right coset reps $R$ of $H$ in $G$.\newline
Task: To find a generating set for H\newline

We are faced with the challenge of controlling the size of the generating set
at each level

\subsection{Controlling the size of the generating set}
$\pi, \psi \in G$
$1^\pi = 1^\psi$

Replace $\{\pi,\psi\}$ with $\{\pi,\pi^{-1} \psi\}$

Ensuring that in our set S there is exactly one element sending 1 to k.

\begin{observation}
\item  By doing this repeatedly for every $\psi$ which maps 1 to k, $\pi$ is the only element that maps 1 to k.
\item  The extra elements that we put in fixes 1.
\item  We did not change $\langle S \rangle$. Note that this is because you can always
create $\pi^{-1}$  from $\pi$ by multiplying by itself until you get 1.
\end{observation}


\subsection{Task Remaining}
Obtain $R$ - the set of coset reps of $G^{(i+1)}$  in $G^i$

Max number of cosets $= n-i$, since $i+1$ can be taken to atmost $n-i$ 
different places.
\[ (i+1)^{G^{(i)}}=\{k\in\{i+1,\ldots , n\} ~|~ \exists g\in G^{(i)}, (i+1)^g = k\} \]

Let $X$ be the orbit of $(i+1)$, $R$ can be computed by obtaining for each $k\in X_i$, an element in $G^{(i)}$  that maps $(i+1)$ to $k$.

\subsection{Claim}
For any $\Sigma\subseteq\Omega$, the generator set of PointStab can be computed
in $poly(\|\Omega\|^{'} )$ time.
\begin{itemize}
\item Apply Schreier's Lemma
\item Reduce
\item Recurse
\end{itemize}
\section{Membership Testing}
\begin{definition}
Input: Generating set of $G\leq S_n: G= \langle A \rangle $ 	$g\in S_n$
Output : Representation in terms of the generating set 
\end{definition}
(This alone is not
possible as we showed yesterday) Representation in terms of any desired
generating set, and the generating set.

\subsection{Tail Recursion Algorithm Z}
\begin{algorithm}
\caption{Algorithm for Membership Testing}\label{orbit}
\label{alg:tail_rec_membership}
\begin{algorithmic}[1]
\item
Input: g, generating set A for $G^i$ and index i.\newline
if g = id return true\newline
$X_i = (i+1)^{G^{(i)}}$\newline
Compute R: The set of distinct coset representatives.\newline
$k = (i+1)^g$ (image of i+1 on action of g)\newline
$if k \notin X_i$ return false\newline
A' = Generating set of $G^{(i+1)}$ (Use Schreier's Lemma)\newline
Reduce the size of B using procedure above\newline
Pick $g_{ik}$ from R (The element that maps i to k)\newline
$g' = g.g_{ik}^{-1}$\newline
return $Z(g', A', i+1)$\newline
\end{algorithmic}
\end{algorithm}

\section{Some other problems}
\subsection{Subgroup Problem}
Given $G \& H$, generated by $A$ and $B$ respectively, test if $H\leq G$?

Algorithm
For each $b\in B$ check if $b\in G$ (membership problem)

\subsection{Group Intersection Problem}
Given $G \& H$ generated by $A \& B$ respectively, find the gen set of $G\cap H$.

\subsubsection{SetStab reduces to GroupInter}
Let $\{Sym_{\Sigma} \times Sym_{\Omega - \Sigma}\} = H$

The generator set for $S_n$ is known: $(1,\ldots,  n)$, $((1, 2), (2, 3), \ldots, (n-1
n)$

If we want to stabilize $\Sigma$, all we need to do is compute $G\cap H$. This intersection gives us the stabilizing subgroup of G.

\Lecture{Jayalal Sarma}{August 21, 2015}{11}{GroupInter to SetStab and Jerrum's Filter}{Samir Otiv}{$\alpha$}

\section{GroupInter to SetStab}
Before we start, let's look at ways of combining groups.

\subsection{Direct Product}
$G*H=\{(g,h) ~|~ g \in G, h \in H\}$

$(\alpha,\beta)^{(g,h)}=(\alpha^g,\beta^h)$

\subsection{The reduction is almost done}
Let $\Sigma=\{(i,i) ~|~ i \in \Omega \}$

\subsection{Claim}
$G \cap H=SetStab(\Sigma)$

\subsection{Proof}
Forward:\newline
	$a \in G \cap H$\newline
	$(a,a) \in S_n*S_n$\newline
	(i,i) $\in \Sigma, (i,i)^{(a,a) } \in \Sigma$\newline
	$\therefore (a,a) \in SetStab(\Sigma)$\newline\newline
Backward:\newline
	$(a,b) \in SetStab(\Sigma)$\newline
	$\therefore \forall (i,i) \in \Sigma:(i,i)^{(a,b)}=(j,j)$\newline
	$j=i^a$\newline
	$j=i^b$\newline
	$\therefore a=b \in$ G,H respectively.\newline
	$\therefore a \in G \cap H$\newline


\section{Jerrum's Filter: Reduce the size of the generating set to n-1}
So far we know a way of constructing it in $n^2$. We're going to take this to the next level.
Another result: Neumann $\lfloor n/2 \rfloor$

\subsubsection{Jerrum's Filter}
S is a given set of permutations\newline
$G = <S>$\newline
G acts on $\Omega$

\subsubsection{Define a graph}
$X_S (V,E)$, where $V=\Omega$\newline
$\forall g \in S$, if $i_g \in \Omega$  is the smallest index moved by g, put an edge $(i_g,i_g^g ) \in E$\newline
View this as an undirected graph.

\subsubsection{Observation}
$g^{-1}$  maps $i_g^g$  to $i_g$

\subsubsection{Measure of weight}
$T \subseteq S_n$
$wt(X_T )=\Sigma _{g \in T}�i_g$

\subsubsection{Crude bound on $wt(X_T )$}
$wt(X_T )\leq \|T\|n$

\subsubsection{The algorithm}
Maintain a set A such that $X_A$  is acyclic. This is the invariant of the algorithm. (If a new element g comes in, we modify the graph to maintain the acyclicness)\newline
Process each $g \in S$ one by one.\newline\newline
For a given g:

	If $g \notin  <A>$, then add g to A and add $e_g$  to the graph $X_A$.

	If there's no cycle created, there's nothing to be done. If there is, consider the cycle created. Let $i_0$  be the least index in the cycle.

	Now, both the edges in the cycle incident to $i_0$ must correspond to $i_0$. (If they didn't then the neighbour of $i_0$ wouldn't be the least moved element by the corresponding group element).
	
	Now consider the walk around the circle from $i_0$ to $i_0$. This corresponds to a product of group elements of the form $g_0 g_1^{\epsilon_1}, \ldots, g_k^{\epsilon_k}$. Now, just get rid of $g_0$, and insert h in its place. This will not change the group generated by S, since reachability is preserved.
	
	Now since $i_0$ is fixed by all of $g_0, g_1,\ldots, g_k$, therefore the least element moved by h is $>$ i. Therefore, the weight of the graph increases on doing this operation.
	
	Therefore, in polynomially many steps, we will hit the upper bound, and will end up with an acyclic graph.
	
	On repeating this with all elements, we have a generating set A' with an acyclic graph $X_{A'}$, and therefore $|A|\leq n-1$.

\end{document}
