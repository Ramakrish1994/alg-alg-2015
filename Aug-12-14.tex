\Lecture{Jayalal Sarma}{August, 12 2015}{7}{Reduction of $\cal{GA}$ to $\cal{GI}$}{Sahil Sharma}


\section{Recap}
Let us denote by $\cal{COLOR}$-$\cal{GI}$ the problem of finding whether there is a coloring preserving isomorphism. 
In the previous few lectures we showed that $\cal{COLOR}-\cal{GI} \le \cal{GI}$.
We also showed that $\cal{GI} \le \cal{GA}$. It was also proved that $\cal{COMPUTE}-\cal{GI} \le \cal{GI}$. In this lecture, we show that $\cal{GA} \le \cal{GI}$. This shows that the graph isomorphism and graph automorphism problems are equivalent.


\section{Towers of Sub-Groups of $G$}
The discussion in this section sets up the notation and concepts related to towers of sub-groups of any given group $G$.
The concept can be applied to any group, but we will use this specifically in the case of $\cal{AUT}$$(X)$, for a given graph $X$. $\cal{AUT}$$(X)$ is defined to be the group of all automorphisms of a given graph $X$.\\
Given a group $G$, we define tower of sub-groups of $G$ to be a sequence of groups 
$\{G^{0}, \cdots, G^{k}\}$ such that $G^{i} \le G^{i-1}	$ for all $i$ and $G^{k} = \{e\}$ is the group
 which contains only the identity element. That is, $G^{i}$ is a sub-group of $G^{i-1}$. 
Hence the sequence of sub-groups looks like $G^{k} = \{e\} \le  G^{1} \le \cdots \le G = G^{0}$. \\\\
There is a coset structure that $G^{i}$ generates in $G^{i-1}$ and we seek to exploit this structure to get a
 $O(n\log n)$ sized generating set of $\cal{AUT}$$(X)$ efficiently, given a $\cal{GI}$ solving procedure. \\
The number of cosets of $G^{i}$ in $G^{i-1}$ is called the {\it index} of $G^{i}$ in $G^{i-1}$.
There is a notion of a {\it coset representative} of a coset, with respect a given group and it's sub-group.
This is just any arbitrarily chosen element of that coset. Suppose the sub-group involved is $G^{i}$,
then for any two elements in the same coset, say $g$ and $g'$ we have $G^{i}.g = G^{i}.g'$, by the definition of the coset (since it can generate the coset).
Hence, any element can be chosen as the representative of the coset.

\section{Unique representation of a group in terms of coset representatives}
\begin{claim}
Consider an element $g \in G^{i-1}$. Also, let the coset in which it lies be $C_{1}$ and the representative for that coset be $u_{1}$. $\exists$ a unique $h_{i} \in G^{i}$ such that 
$$g = h_{i}.u_{1}$$.
\end{claim}
\begin{proof}
$g \in C_{1} \implies \exists h_{i} \in G^{i}$ such that $g = h_{i}.u_{1}$.\\
Such a $h_{i}$ is unique since $h_{i} = g.u_{1}^{-1} = h_{i}'$. 
\end{proof}
\begin{theorem}
Suppose we fix the tower of sub-groups of $G$ as well as the co-set representatives for each
 consequetive pair of $\{$group, sub-group$\}$ in the tower of sub-groups. We claim that each element of the original group $G$ can be represented as a product of some elements of this set of coset representatives.
\end{theorem}
\begin{proof}

By the above claim, we know that such an $h_{1}$ exists for any element $g \in G$. Now we ask, which coset does $h_{1}$ belong to, in the set of cosets generated by $G_{2}$ in $G_{1}$. By the application of the same claim to $h_{1}$ and the pair of groups $(G_{1}, G_{2})$ we get a unique $h_{2}$ such that $h_{1} = h_{2}.u_{2}$, depending on the coset to which $h_{1}$ belongs. Continuing in this way, as we go down the tower of subgroups we end up with a unique representation for the element we started off with, that is, $g$. Note that the representation is unique since at each stage the $h_{i}$ were found in a unique way, and since the coset-representatives are held fixed.
\end{proof}

\section{Finding a generating set efficiently, for $\cal{AUT}$$(X)$}
To find a generating set for $\cal{AUT}$$(X)$ it suffices to find a tower of sub-groups of $\cal{AUT}$$(X)$ and the 
coset representatives at each level. This is because according to the above theorem, we can represent each element of $\cal{AUT}$$(X)$ uniquely as a product (composition) of a sequence off coset representatives.\\
Let us define the tower of sub-groups in such a way that computing the coset-representatives becomes efficient (using $\cal{GI}$).\\\\
The sub-groups are defined as :
$$G^{i+1} := \{ g \in G^{i} | 1^{g} = 1\}$$
That is, $G^{i}$ is the sub-group of automorphisms which maps all the nodes of the graph in the range $\{1, \cdots, i\}$ to themselves. We can check that $G^{i}$ is indeed a group since the composition of two permutations which maps all the nodes of the graph in the range $\{1, \cdots, i\}$ to themselves also does the same. Moreover, the identity permutation is the identity for this sub-group too, and the inverse permutation is defined in the standard way and has the same properties as the inverse permutation in $\cal{AUT}$$(X)$. Hence, this is indeed a sub-group.\\\\
Let the number of cosets generated by $G^{i+1}$ in $G^{i}$ be $l_{i}$.
\begin{claim}
$$l_{i} \le n-i$$\end{claim}
\begin{proof}
Consider an automorphism $g \in G^{i}$. This maps the element $i+1$ to some element in the range $\{i+1, \cdots, n\}$. This is because the other elements are already fixed. Let $k = (i+1)^{g}$ be the element to which $i+1$ is mapped and $l$ be the element such that $l^{g}= i+1$. Then consider the permutation $g'$ which retains other mappings from $g$ but changes the above two mappings to $(i+1)^{g'} = i+1$ and $l^{g'} = k$. Note that $g'$ is also an automorphism since $l$ being mapped to $i+1$ implies that if $l$ is mapped to the image of $i$ the automorphism would still be preserved (adjacency and non-adjacency is preserved). Also note that $g'$ maps $\{1, \cdots , i+1\}$ and hence is in $G^{i+1}$. This means that whatever is the number of automorphisms in $G^{i}$ cannot exceed the number of automorphisms in $G^{i+1}$ multiplied by $n-i$ since the $i+1$ can potentially map to only $n-i$ elements.
\end{proof}\\
The algorithm for finding the coset representatives is now simple. For all $i$, look for automorphisms which map all of $\{1, \cdots, i\}$ to themselves and map $i+1$ to each element in the set $\{i+1, \cdots,n\}$, one by one and ask the question: Is there an automorphism which preserves these mappings. This is done by introducing an instance of the $\cal{COLOR-GI}$ and solving it using $\cal{GI}$ as demonstrated in the previous lectures. Note that this can be done since the elements from  $\{1, \cdots, i\}$ 
can be colored with the color which is the number itself, in both the graphs $X_1$ and $X_2$. $i+1$ in set $X_1$ can be colored with the color $i+1$ and $j$ from $X_2$ is also colored with $i+1$. Rest of the elements are colored with a new color $i+2$. In this way, we can cast the problem as a $\cal{COLOR-GI}$ and solve it using $\cal{GI}$. Each time we get an answer as yes, we then use the reduction $\cal{COMPUTE-GI} \le \cal{GI}$ to find the actual permutation. This permutation is one of the coset representatives and hence in this manner we get each coset representative. Note that for each search for coset representative we make only polynomially many calls to $\cal{GI}$ and the number of such representatives is upper bounded by $n^2$ and hence our reduction is polynomial time and computes a generating set of $\cal{AUT}$$(X)$ in polynomial time, given that we can solve $\cal{GI}$ in polynomial time.\\\\
\begin{definition}
	Strong Generating Set: A generating set for $\cal{AUT}$$(X)$ obtained in the manner above making use of coset representatives and tower of sub-groups is known as a strong generating set.
\end{definition}




\Lecture{Jayalal Sarma}{August, 14 2015}{8}{Generalization of group-theoretic problems}{Sahil Sharma}
\section {Recap}
In the previous lecture we showed that $\cal{GA} \le \cal{GI}$. We also defined certain generic group theoretic concepts like {\it tower of sub-groups}, {\it coset representatives} and the notion of {\it strong generating sets}.
\\ Also note that the reduction $\cal{\#GA} \le \cal{GI}$ can be done using the set of reductions ${\cal{\#GA} \le \cal{GA} \le \cal{GI}}$ where the first reduction follows from the fact that the generating set of $\cal{AUT}$$(X)$ was in fact a strong generating set and hence each element of $G$ could be obtained uniquely by composing elements of the generating set. This would imply that the size of the automorphism group is $\displaystyle \prod_{i} l_{i}$} where $l_i$ was defined in the last lecture.\\\\
Abstracting the sub-problems which were encountered in doing the previous reduction, we get a set of $4$ related group-theoretic problems listed below.
\section{Problem 1}
Given a generating set for $G$, can we compute the order of the group, that is, the number of elements in the group.\\\\
One of the things we will try to do, to solve this problem is to convert the given generating set into a strong generating set, since we already know how to solve the problem once that is done.
\section{Problem 2}

Given a group $G$ and it's sub-group $H$ through respective generating sets, we want to check whether an arbitrary $g \in G$ belongs to $H$.\\\\
Note that problem 2 can be solved using problem 1. This is because we could just add $g$ to the generating set $S$ of $H$ and compare the sizes of the groups generated by $S$ and $S \cup \{g\}$  . If the sizes are unequal we conclude that $g$ was not in $H$. This is because if $g$ were in $H$, then the generating set $S$ could have generated $g$ as well and adding it to $S$ could not have resulted in any new elements. 
\section{Problem 3}
{\bf Orbit Computation} : Given a group $G \le S_{n}$ (via its generating set $S$), compute the orbits of the action of $G$ on $[n]$.\\\\
We show a different way of viewing a group wherein it will be clear that any generic group can be looked at as a permutation group. Consider an element $g \in G$. Consider any arbitrary ordering of the elements in $G$. Then the multiplication $G.g$ sends the elements of $G$ to a permutation of themselves. Hence, with each element 
$g \in G$, we can associate a permutation of the elements of $G$. If the order of the group, $|G| = k$, then the resulting group of permutations would be a sub-group of $S_{k}$. This defines the notion of a group acting on itself. Note that it is clear that the resulting set of permutations is a group since multiplication in original group translates to composition in the new group and the identity element in the original group is associated with the identity permutation and so on.\\\\
We know, due to the orbit stabilizer lemma that for any $\alpha \in  G$
$$|\alpha ^{G}|*|G_{\alpha}| = |G|$$
This equation alongwith Problem 3 can be used for solving Problem 1 since if we can copute the orbit size of $\alpha$ then we can recurse for the sub-group $G_{\alpha}$. 
\section{Problem 4}
We know that $\cal{AUT}$$(X)$ acts on $[n]$. This can also be visualized as $\cal{AUT}$$(X)$ acting on the set of edges $E(X)$. This is because the automorphisms map edges to edges and non-edges to non-edges. Hence, any edge of the graph gets maped to another edge. Hence we can think of $\cal{AUT}$$(X)$ as acting on the set of edges.\\\\
Extending this idea, we can also think of $S_n$ which is a super-group of $\cal{AUT}$$(X)$ as acting on the set of all potential edges given by $K = \{ \{i, j\} | i, j \in [n]$ and $i \le j\}$. 

\begin{definition}
	Set Stabilizer of \Sigma:
	\begin{center}
	$\cal{SETSTAB}$$(\Sigma) := \{ g \in G | \Sigma^{g} = \Sigma\}$
	\end{center}
\end{definition}
Given a group $G \le S_n$ which acts on a set $\Omega$, we define the Set Stabilizer of $\Sigma \subseteq \Omega$ as the set of all permutations in $G$ which map elements of $\Sigma$ to elements of $\Sigma$ and all non elements of $\Sigma$ to non-elements of $\Sigma$. Note that $\cal{SETSTAB}$$(\Sigma)$ is a sub-group of $G$. In the case of automorphism groups, the mapping is $G = S_n$, $\Omega = K$, $\Sigma = E(X)$ and $\cal{SETSTAB}$$(\Sigma)$$=$$\cal{AUT}$$(X)$.\\\\
Having set up all the notation, let us state the computational question we seek to answer:\\\\
Given a group $G$ via its generating set $S$, a set $\Omega$ on which it acts and a subset $\Sigma \subseteq \Omega$, output a generating set for $\cal{SETSTAB}$$(\Sigma)$.

\section{Solution to Problem 3}
For completeness's sake, let us restate the question.\\
{\bf Orbit Computation} : Given a group $G \le S_{n}$ (via its generating set $S$), compute the orbits of the action of $G$ on $[n]$.\\\\
We can see that the orbits of the action of $G$ on $[n]$ partition the set into different sub-sets. Hence, what we would want to do is to compute the partition. We try to cast this problem as a graph-theoretic problem.
The vertices are the numbers in the set $[n]$.
If $j \in [n]$ lies in the orbit of $i \in [n]$, then there is an edge from $i$ to $j$. We know due to the manner in which orbits are constructed and the fact that they induce partitions that the final graph looks like a set of cliques. Hence, we can try to compute the graph partitially (by finding some edges) and make use of transitive closures for finding the rest of the edges in the graph until we reach a fixed point, at which point we can say that we have computed the partition and hence the orbits.  
