\Lecture{Jayalal Sarma}{August, 24 2015}{13}{Group Intersection Problem under Special Cases-I}{Monosij Maitra}{$\alpha$}

\section{Recap}
In the previous lecture, we looked at \emph{Jernim's filter} algorithm to reduce the size of the generating set $\langle S\rangle$ of a subgroup $G$ of $S_n$ to at most $n-1$. Thereafter we looked at the \emph{Group Intersection Problem} which is as follows:
\begin{problem}[\textbf{Group Intersection}]\label{1}
	Given two groups $G$ and $H$ via their generating sets $\langle A\rangle$ and $\langle B\rangle$ respectively, with $G\leq S_n$ and $H\leq S_n$, can we compute the generating set for the group $G\cap H$?
\end{problem}
We also saw that the \emph{Group Intersection Problem} can be reduced to the \emph{Set Stabilizer Problem} and vice-versa. But both are  harder than \emph{Graph Automorphism Problem} as in we know of the fact that $GA\leq SetStab\equiv\text{ Group Intersection Problem}$. In this lecture we are interested in solving Problem \ref{1} in a special setting of \emph{Normal Subgroups}. But before that let's review the basics of \emph{Normal Subgroups}.

\section{Normal Subgroups}
We already know that if $H$ and $G$ are two groups and $H\leq G$, then the right cosets of $H$ in $G$ are given by $Hg = \{hg~|~h\in H\text{ and }g\in G\}$. Similarly the left cosets are given by $gH = \{gh~|~h\in H\text{ and }g\in G\}$. By \emph{Lagrange's Theorem} we also know that any two right cosets of a subgroup $H$ are either \emph{disjoint} or \emph{equal} i.e. for any $g_1,g_2\in G$ with $g_1\neq g_2$, either $Hg_1\cap Hg_2 = \emptyset$ or $Hg_1 = Hg_2$. The same conditions hold true for the left cosets of $H$ too.\\\\In a general setting for $H\leq G$ it may not be the case that $Hg = gH$. But when this happens, some special structures follow automatically from the context. In that context we define the following:
\begin{definition}
	Let $H\leq G$ such that $\forall g\in G,~Hg = gH.$ Then $H$ is called a {\bf Normal Subgroup} of $G$ and is denoted by $H\triangleleft G$.
\end{definition}
One special property of such a subgroup $H$ follows directly from the definition: The set of all left cosets and right cosets remain the same. Another special property of $H$ is that the set of all cosets \emph{follows a group structure} with respect to the following operation $*$ defined as $Hg_1*Hg_2 = Hg_1g_2~\forall g_1,g_2\in G$. The proof of the this fact is an easy argument since the set of cosets can be verified easily to follow the four properties of any group as follows:
\begin{itemize}
	\item {\bf Closure} Since $G$ has closure property and $Hg_1*Hg_2 = Hg_1g_2$, hence this itself implies closure of the set of cosets under $*$.
	\item {\bf Existence of Identity} The subgroup $H$ itself serves as the identity element in the set since $\forall g\in G,~Hg*H = Hg$.
	\item {\bf Existence of Inverse} Since $\forall g\in G,~Hg*Hg^{-1} = Hgg^{-1} = H$, therefore inverse exists.
	\item {\bf Associativity} Follows easily from the definition.
\end{itemize}
The set of all cosets of a normal subgroup $H$ of $G$ has a distinct name in the literature of Group theory. It is defined as follows:
\begin{definition}\label{QG}
	Let $H\triangleleft G$. The set of all cosets $\{Hg~|~g\in G, h\in H\}$ under the binary operation $*$ is called the {\bf Quotient Group} of $G$ by $H$.
\end{definition}
We know that if $g'\in Hg$, then $Hg' = Hg$ i.e. $Hg$ and $Hg'$ refers to the same coset. We want to claim the following:
\begin{claim}
	Let $g_1'\in Hg_1,~g_2'\in Hg_2$ and $H\triangleleft G$. $Hg_1'*Hg_2'=Hg_1'g_2'$ and $Hg_1*Hg_2 = Hg_1g_2 \Longrightarrow Hg_1'g_2' = Hg_1g_2$. 
\end{claim}
\begin{proof}
	\begin{eqnarray}
		Hg_1g_2 &=& Hg_1'g_2~~~\text{[Since }g_1'\in Hg_1\text{]}\\
				&=& g_1'Hg_2~~~\text{[Since }H\triangleleft G\text{]}\\
				&=& g_1'Hg_2'~~~\text{[Since }g_2'\in Hg_2\text{]}\\
				&=& Hg_1'g_2'~~~\text{[Since }H\triangleleft G\text{]}
	\end{eqnarray}
\end{proof}
We want to emphasize two things here: firstly, the above claim shows that $*$ is well-defined whenever $H\triangleleft G$. Secondly, note that this claim holds only when $H\triangleleft G$. Otherwise, the operation $*$ on the set of cosets is in general not well-defined. However, the other direction of the above claim also holds as shown by the following claim.
\begin{claim}
	If $H\leq G$, then $\forall g_1,g_2\in G,~Hg_1*Hg_2 = Hg_1g_2~\Longrightarrow H\triangleleft G$. 
\end{claim}
\begin{proof}
	It suffices to argue that $\forall g\in G,~ gHg^{-1}\subseteq H\Longleftrightarrow Hg=gH$ i.e. $H\triangleleft G$, by definition. We observe the following: $\forall g\in G,$
	\begin{eqnarray*}
		gHg^{-1}&\subseteq& gHg^{-1}H\text{ [since }e\in H]\\
				&=&(gH)(g^{-1}H)\text{ [By Associativity]}\\
				&=&gg^{-1}H\text{ [By hypothesis]}\\
				&=&eH\\
				&=&H
	\end{eqnarray*}
	Thus $gHg^{-1}\subseteq H \Longleftrightarrow \exists h' \in H$ such that $ghg^{-1} = h'\text{ i.e. }gh = h'g \Longleftrightarrow gH = Hg \Longleftrightarrow H\triangleleft G$.
\end{proof}
Having geared up with the basic notions of Normal Subgroups, now we can ask the following computational question.
\begin{problem}[\textbf{Group Normality}]\label{2}
	Given two groups $G$ and $H$ via their generating sets $\langle A\rangle$ and $\langle B\rangle$ respectively, can we check if $H\triangleleft G$?
\end{problem}
\subsection{Normalizer and Normal Closure of a Group}
Suppose for some groups $H$ and $G$ where $H\leq G$, $H\ntriangleleft G$. This happens because there is at least one $g\in G$ such that $Hg\neq gH$. In this context observe the set of all $g\in G$ such that they make $H$ \emph{look like} a normal subgroup in $G$. In other words, collect all such $g\in G$ such that $Hg = gH$ is true. This leads us to the following definition in the literature of Group theory again.
\begin{definition}\label{normalizer}
	The set $N_G(H) = \{g\in G~|~Hg = gH\}$ is called the {\bf Normalizer} of $H$ in $G$.
\end{definition}
Observe that for any $H\leq G$, $N_G(H) \neq \emptyset$. More concretely speaking $H\in N_G(H)$. This follows easily since $h\in H\Longrightarrow h\in G\text{ and }hH = Hh = H$. Therefore $N_G(H)$ is at least containing the set $H$ itself. Another interpretation is that $|H|\leq |N_G(H)| \leq |G|$(since every group $G$ is a normal subgroup of itself trivially.) Once we are comfortable with the definition of $N_G(H)$, an easy observation is that the algebraic structure $(N_G(H),*)$ is again a group under the operation $*$ defined in Definition \ref{QG}. One more computational question arises in this context.
\begin{problem}[\textbf{Normalizer Computation}]\label{3}
	Given two groups $G$ and $H$ via their generating sets $\langle A\rangle$ and $\langle B\rangle$ respectively, output the generating set of $N_G(H)$.
\end{problem}
Clearly, the \emph{normalizer} gives the \emph{largest subgroup} of $G$ in which $H$ is a normal subgroup. Then a natural question arises about seeking the smallest normal subgroup of $G$ that contains $H$ in it. Thus the following definition comes naturally.
\begin{definition}
	The smallest normal subgroup of $G$ that contains $H$ is called the {\bf Normal Closure} of $H$ in $G$.
\end{definition}
Keeping the above definition in our mind another question pops up naturally regarding the computability of normal closure of a group as defined formally below.
\begin{problem}[\textbf{Normal Closure Computation}]\label{4}
	Can we give an efficient algorithm for computing the generating set of the normal closure of a subgroup $H$ of $G$? 
\end{problem}
\jsay{To think on this, how to come up with such an algorithm for a subgroup $H$?}

\section{Group Intersection Problem in the Context of Normalizers}
Now let us get back to the source of our original motivation which is the Permutation group $S_n$. As usual we take $G\leq S_n$, $H\leq S_n$. By Definition \ref{normalizer}, $N_{S_{n}}(H)$ is the normalizer of $H$ in $S_n$. We consider \emph{a special case when} $G$ \emph{normalizes} $H$ i.e. $\forall g\in G, Hg = gH$. This means $G\leq N_{S_{n}}(H)$.\\\\We are already aware of the fact that for general groups $G$ and $H$, we do not have an efficient algorithm for computing the generating set $\langle S\rangle$ of $G\cap H$. Before getting on to the actual algorithm for computing the generating set of $G$ and $H$ in the special setting defined above let us have some more observations. Observe the set $GH = \{gh~|~g\in G, h\in H\}$. $GH$ forms a group again since
\begin{itemize}
	\item {\bf Closure} If $g_1h_1,g_2h_2\in G$, then
	\begin{eqnarray*}
		g_1h_1.g_2h_2 &=& g_1g_2.h_1'h_2~~ [\text{Since }G\text{ normalizes }H]\\
		&=&gh\in GH.
	\end{eqnarray*}
	\item {\bf Existence of Identity} Setting $g = h = e$ it is trivial to observe that $e\in GH$.
	\item {\bf Existence of Inverse} $\forall gh\in GH,~ h^{-1}g^{-1}\in GH$. Thus,
	\begin{eqnarray*}
		gh.(h^{-1}g^{-1}) &=& g(hh^{-1})g^{-1}\\
						  &=& g(e)g^{-1}\\
						  &=& gg^{-1}\\
						  &=& e\in GH.
	\end{eqnarray*} 
	\item {\bf Associativity} Follows naturally from the definition.
\end{itemize}
Another observation is that for our setting where $G$ and $H$ are subgroups of $S_n$, if $G$ normalizes $H$ and since $GH$ is a group itself, then $GH\leq S_n$. This may not hold in general though. We are now just a few steps away from describing the algorithm for Problem \ref{1}. Although the algorithm will be given explicitly in the next lecture, the first idea is again to observe the set $G^{(1)}$ which fixes the element $1$. By the previous observation just described above, we can infer that $G^{(1)}H\leq S_n$, since any subgroup of $G$ will also normalize $H$. In general while constructing the set $G^{(i)}$ which fixes $1,2,\ldots,i$, $G^{(i)}H\leq S_n$. Now we are ready to describe the construction of the \emph{Tower of Subgroups} as follows:
\begin{eqnarray*}
	G\cap G^{(0)}H \geq G\cap G^{(1)}H \geq G\cap G^{(2)}H \geq \ldots \geq G\cap G^{(n-1)}H
\end{eqnarray*}
Two quick observations are:\\\\{\bf Observation 1.} On the right end of the above chain $G^{(n-1)}$ is the subgroup which \emph{fixes} all the elements $1,2,\ldots,n-1$. Hence it gives the identity permutation in $S_n$ exactly. Thus $G\cap G^{(n-1)}H  = G\cap H$.\\\\{\bf Observation 2.} On the left end of the above chain $G\cap G^{(0)}H = G\cap GH = G$. This is because $GH = G$ as $H$ is contained in $G$ as we proved earlier and $G\leq S_n$. Hence $G\cap GH = G$.\\\\Thus the above tower of subgroups really look like that as shown below.
\begin{eqnarray*}
	G\geq G\cap G^{(1)}H \geq G\cap G^{(2)}H \geq \ldots \geq G\cap H
\end{eqnarray*}


%==========================================================================================================================================


\Lecture{Jayalal Sarma}{August, 25 2015}{14}{Group Intersection Problem under Special Cases-II}{Monosij Maitra}{$\alpha$}
\section{Recap}
In the previous lecture, we were devising the tools that we need for solving Problem \ref{1} under the special setting of a group $G$ acting as a {\bf normalizer} for group $H$.. We discussed {\bf Normal Subgroups} along with the concepts of {\bf Normalizers} and {\bf Normal Closure} in this context also. Then we looked at the \emph{Tower of Subgroups} that we are going to use to give the algorithm for Problem \ref{1}. The \emph{Tower of Subgroups} looked like the following:
\begin{eqnarray*}
	G\geq G\cap G^{(1)}H \geq G\cap G^{(2)}H \geq \ldots \geq G\cap H
\end{eqnarray*}
\section{Getting to the Algorithm }
In the above chain of subgroups we are already given the generating set of $G$. We can apply {\bf Schreier's Lemma} to this chain of subgroups to get the generating set of $G\cap H$. But before doing that we must ensure of three things:
\begin{enumerate}
	\item Computation of the coset representatives of the subgroups at each level in the chain.
	\item Testing membership in the subgroups.
	\item Ensuring that the number of cosets at all level must be small.
\end{enumerate}
These three properties must be satisfied in order to get a \emph{strong generating set}. 
\begin{itemize}
	\item We can apply our known algorithm to find the set of coset representatives of the subgroup $G^{(i)}H$ in $G^{(i-1)}H$.
	\item For the second property, since we already have a generating set $\langle A\rangle$ for $G$ as input, we have to find the generating set of $G^{(1)}H$ such that the membership testing of elements of the subgroup can be done efficiently. Now $G^{(1)}H$ is defined as the product of all the elements in $G^{(1)}$ and $H$. Hence the generating set of $G^{(1)}H$ is the \emph{union} of the generating set $S_1$ of $G^{(1)}$ and that of $H$ i.e. $\langle S_1 \rangle\cup \langle B \rangle$. Once we get these, we can easily apply our \emph{Membership Testing algorithm} which was done in a previous lecture.
	\item For an assurance of the third property we need to prove the following lemma.
\end{itemize}
\begin{claim}\label{coset}
	The index of any subgroup in its parent group i.e. $[G\cap G^{(i)}H~:~G\cap G^{(i-1)}H]\leq (n-i)$.
\end{claim}
The crux of the proof of the above claim lies in the following arguments about how a coset of $G^{(i)}H$ actually looks like which is inside $G^{(i-1)}H$. In other words, $G^{(i)}Hg$ is a coset of $G^{(i)}H$ inside $G^{(i-1)}H$ if and only if $g\in G^{(i-1)}H$. This implies that $g = g'h'$, where $g'\in G^{(i-1)}$ and $h'\in H$. Thus we see that 
\begin{eqnarray*}
	G^{(i)}Hg &=& G^{(i)}Hg'h'\\
			  &=& G^{(i)}g'Hh'~~[\text{Since }G^{(i-1)}\text{ normalizes }H]\\
			  &=& G^{(i)}g'H~~[\text{Since }Hh' = H]\\
			  &=& (G^{(i)}g')H.
\end{eqnarray*}
The rest of the proof is left as an exercise. Performing an intersection with $G$ changes the cosets but we need to argue that the \emph{number of cosets} still remains the same and is essentially bounded above by $(n-i)$ in the $i$-$th$ level of the tower of subgroups. As an intermediate observation, we claim the following:
\begin{claim}
	$H\triangleleft GH$.
\end{claim}
\begin{proof}
	Let $g'\in GH$. Then $g' = gh$ for some $g\in G, h\in H$. For deducing $H\triangleleft GH$, we need to show that $\forall g'\in GH, Hg' = g'H$. So we proceed as follows:
	\begin{eqnarray*}
		Hg' &=& Hgh\\
			&=& gHh~~[\text{Since }G\text{ normalizes }H]\\
			&=& ghH~~[\text{Since }Hh = hH]\\
			&=& g'H.
	\end{eqnarray*}
\end{proof}
\subsection{The Algorithm}
Now we are in a perfect set up to describe the algorithm for computing the generating set of $G\cap H$. The algorithm just applies {\bf Schreier's Lemma} \emph{recursively} at all the $(n-1)$ levels in the tower of subgroups to compute the generating set of $G\cap G^{(i)}H,~\forall i=1,2,3,\ldots,(n-1)$.\\\\We know that the number of cosets in the $i$-$th$ level is bounded above by a small quantity $(n-i)$ as Claim \ref{coset} shows. From the REDUCE algorithm that we saw in a previous lecture we know that the size of the generating set at each of the $(n-1)$ levels of the tower of subgroups computed by {\bf Schreier's Lemma} is $O(n^2)$. Therefore the asymptotic running time of this algorithm is $O(n^3)$.

\section{Another Special Case of Group Intersection Problem}
In this section we are going to solve the Problem \ref{1} in another special setting. We have already mentioned of the fact that $GA\leq SetStab\equiv\text{ Group Intersection Problem}$. We are also aware that there are no efficient efficient algorithms for Graph Isomorphism problem in general. In this context we have seen that $\mathcal{COLOR-GI}\leq \mathcal{GA}$. However, the hard case of this scenario arises when the number of colours, $k=1$. In that case it gets mapped to the same problem of Graph Isomorphism. Thus instead of getting our attention to the \emph{number of colours}, we focus on the size of each colour class, i.e. $\psi^{-1}(i), i\in \{1,2,\ldots,k\}$ as per our notations discussed earlier. We assume that $\forall i\in\{1,2,\ldots,k\}, |\psi^{-1}(i)|\leq b, b\in \mathbb{N}$ i.e. the size of each colour class in bounded by a {\bf \emph{small constant}} $b$ ($b$ is independent of the number of vertices, $n$ in the corresponding graph). Now the question we ask is the following:
\begin{problem}
	Given two graphs $\mathcal{X}_1(V_1,E_2)$ and $\mathcal{X}_2(V_2,E_2)$ with the above setting, can we check if they are isomorphic efficiently?
\end{problem}
Since we know that checking Isomorphism between $\mathcal{X}_1$ and $\mathcal{X}_2$ can be reduced to checking Automorphism group of $\mathcal{X}(V,E)=\mathcal{X}_1\cup \mathcal{X}_2$. This additional constraint of colour imposes  an association between an edge in the graph with another one whose vertices are of same colour so that it becomes a \emph{colour preserving map}. In other words,
\begin{equation*}
	Aut(\mathcal{X})\leq Sym(C_1)\times Sym(C_2)\times Sym(C_2)\times \ldots \times Sym(C_k)
\end{equation*}
where $k =$ number of colours and $C_i = \psi^{-1}(i), \forall i\in \{1,2,\ldots,k\}$. We must note that all elements in the right hand side of the above may not be elements in $Aut(\mathcal{X})$. Thus we have shown that $\mathcal{COLOR-GI}\leq \mathcal{COLOR-GA}$. Now the idea is to use the known $SetStab$ reduction so that we can compute the stabilizer set efficiently. From that context we denote $E_i = E\cap \dbinom{C_i}{2}$ to be the set of all {\bf \emph{intracolour edges}} for each colour $i$ and $E_{ij} = E\cap (C_i\times C_j)$ to be the set of all {\bf \emph{intercolour edges}} between any two \emph{distinct} colour classes $i$ and $j$. Note that an edge inside $E_i$ is mapped within $E_i$ again and an edge inside $E_{ij}$ is mapped within itself too, thereby obtaining a \emph{colour preserving map}. In other words, any permutation $g \in Sym(C_i)$ is an automorphism if and only if $E_i^g = E_i$ and $E_{ij}^g = E_{ij}$ i.e. we have got the desired $SetStab$ form.\\\\Now we view this with respect to the $PointStab$ form by taking all the $E_i$ and $E_{ij}$ as points. Thus we view $S_n$ as acting on the following $\Omega$ given by
\begin{equation*}
\Omega = \left(\bigcup_iC_i\right)\cup \left(\bigcup_i2^{\dbinom{C_i}{2}}\right) \cup\left(\bigcup_i2^{C_i\times C_j}\right)
\end{equation*}
where the term $2^{\dbinom{C_i}{2}}$ gives the potential subsets of all \emph{intracolour edges} while the term $2^{(C_i\times C_j)}$ gives the potential subsets of all \emph{intercolour edges}. Now we bound the size of $\Omega$ as follows:
\begin{equation*}
|\Omega| \leq n + k.2^{\dbinom{b}{2}} + \dbinom{b}{2}.2^{b^2}
\end{equation*}
Since $b$ is a small constant thus $|\Omega|$ is polynomial in $n$.
