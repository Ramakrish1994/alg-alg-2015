\Lecture{Jayalal Sarma}{Aug 5, 2015}{3}{Algebraic Approach to finding Perfect Matchings in Graphs}{K Dinesh}{$\delta$}
\noindent

\section{Application to Graph algorithms}
Consider the following problem of finding perfect matching.
\begin{definition}[Finding Perfect Matching]
Given a bipartite graph $G(V_1,V_2, E)$, we need to come up with an $E'
\subseteq E$ such that $\forall u \in V_1 \cup V_2$, there is exactly one edge
incident to it in $E'$.
\end{definition}

We shall give a polynomial formulation for the problem. Given $G(V_1, V_2, E)$
with vertex sets $|V_1| = |V_2| = n$. Define an $n\times n$ matrix $A$ where
$A(i,j) = 1$ if $(i,j) \in E$ and is $0$ otherwise for all $(i,j) \in V_1 \times
V_2$. Recall the determinant of $A$ given
by
\[ det(A) = \sum_{\sigma \in S_n} sign(\sigma) \prod_{i=1}^n A_{i,\sigma(i)}
\]

where \[
sign(\sigma) = \begin{cases} 
	  -1  & \text{ if } inv(\sigma) \text{ is even} \\
	  1 & \text{ if } inv(\sigma) \text{ is odd}
\end{cases}\]
and $inv(\sigma)$ is defined as 
$\left |\{(i,j) ~|~ i < j \text{ and } \sigma(i) > \sigma(j), 1 \le i<j \le n\}
\right |$.
We denote $f(x) \equiv 0$ to denote that polynomial $f(x)$ is the zero
polynomial.
\begin{lemma}
For the matrix $A$ as defined as before, $det(A) \not \equiv 0 \implies
\text{$G$ has a perfect matching}$ \label{lem:det_pm_simple}
\end{lemma}
\begin{proof}
Let $det(A) \not \equiv 0$. Hence there exists a $\sigma \in
S_n$ such that $\prod_{i=1}^n A_{i, \sigma(i)} \ne 0$. Hence the edge set
$E' = \{(i, \sigma(i)) ~|~ 1\le i\le n\}$ exists in $G$ and since $\sigma(i) =
\sigma(j)$ iff $i=j$ for every $i,j$, $E'$ form a perfect matching.

\end{proof}
Note that converse of this statement is not true. For example, consider the
bipartite graph whose $A$ matrix is $A = 
\begin{bmatrix}
		1 & 1 & 0  \\
		1 & 1 & 0  \\
		0 & 0 & 1  
\end{bmatrix}
$. It can be verified that $det(A) = 0$ but the bipartite graph associated has
a perfect matching.

Hence the next natural question, similar to our primality testing problem, is 
to ask for some kind of modification so that 
converse of previous lemma (lemma~\ref{lem:det_pm_simple}) is true. In the
example considered, there were two perfect matchings in $G$ having opposite 
sign due to which the determinant became $0$. So the modification should
ensure that perfect matchings of opposite signs does not cancel off in the
determinant.

One way to achieve this is as follows.  Define a matrix $T$ as
\[ T(i,j) = \begin{cases}
		x_{ij} & (i, j) \in E \\
		      0 & \text{ otherwise}
	\end{cases}
\]
where $(i,j) \in V_1 \times V_2$. This matrix is called as the Tutte matrix.
Now, if we consider determinant of this matrix, we can see that the monomials
corresponding a $\sigma \in S_n$ can be a product of at most $n$ variables.
Hence $det(T)$ is a polynomial in $n^2$ variables with degree at most $n$.

Also in the expansion of determinant, we can observe that each term picks
exactly one entry from every row and column meaning each of the entries 
picked is from a distinct row and column. Hence each of the them is a
bijection and hence is a permutation on $n$ elements. Also corresponding to any
permutation on $n$ elements, we can get a term. This gives us the following
observation.

\begin{observation}
	Set of monomials in $det(T)$ is in one-one correspondence with set of
	all permutations on $n$.
\end{observation}


% TODO do 3x3 det example.

We now give polynomial formulation for the problem of checking perfect
matching in a bipartite graph.
\begin{claim}
For the matrix $T$ as defined before, $det(T) \not \equiv 0 \iff \text{$G$
has a perfect matching}$
\end{claim}
\begin{proof}
	By $det(T) \equiv 0$, we mean that the polynomial $det(T)$ has all
	coefficients as zero. 
($\Rightarrow$) Since $det(T) \not \equiv 0$, there exists a $\sigma \in S_n$
such that the monomial corresponding is non-zero and by definition of
determinant it must be expressed as product of some $n$ set of variables
$\prod_i T(i,\sigma(i))$. The variable indices gives a matching and since
there are $n$ variables this is a perfect matching.

($\Leftarrow$) Suppose $G$ has a perfect matching given by a $M$. Let $\tau$ 
denote the permutation corresponding to $M$. We now
need to show that $det(T) \not \equiv 0$. To prove this, it suffices to show
that there is a substitution to $det(T)$ which evaluates to a non-zero value. 

Consider the following assignment, $\forall~i,j$ 
\[ a_{ij} = \begin{cases}
		1 & \text{ if } (i,j) \in M \\
		0 & \text{ otherwise}
	\end{cases} 
\]
From the formula of determinant, 
\begin{align*}
	det(T) & = \sum_{\sigma \in S_n} sign(\sigma) \prod_{i=1}^n T_{i,
	\sigma(i)}  \\
	& = sign(\tau)\prod_{i=1}^n A_{i, \tau(i)} + \sum_{\sigma \in S_n
	\setminus \{\tau\}} sign(\sigma)\prod_{i=1}^n T_{i, \sigma(i)}
\end{align*}
Now substituting $x_{ij} = a_{ij}$, we get that the first term evaluates to
$sign(\tau)$ since all the entires are $1$. The second term evaluates to $0$
since for all $\sigma \ne \tau$, there must be a $j$ such that $\sigma(j) \ne
\tau(j)$. Hence it must be that $a_{j, \sigma(j)} = 0$ and hence the product
corresponding to $\sigma$ goes to $0$.
\end{proof}

Hence to check if the bipartite graph $G$ has a perfect matching or not, it
suffices to check if the polynomial $det(T)$ is identically zero or not.
Checking if a polynomial is identically zero or not is one of the fundamental
question in this area.

Note that this problem becomes easy if the polynomial is given as sum of 
monomial form. In most of the cases, the polynomial will not be given this way.
For example if we consider our problem, we are just given the $T$ matrix and
$det(T)$ is the required polynomial. Trying to expand $det(T)$ and simplifying
will involve dealing with $n!$ monomials which is not feasible.

Hence the computational question again boils down to checking if a polynomial
is identically zero or not.
%So far we have spoken about finding perfect matching in bipartite graphs. A
%similar result holds for general graphs also. 
%\begin{theorem}
%Given a graph $G(V,E)$ and for $i, j \in V$, $i \ne j$, define an $n \times n$
%matrix $T$ as
%\[T_{i,j} = \begin{cases}
%		x_{ij} & (i,j) \in E, i < j\\
%	       -x_{ji} & (i,j) \in E, i > j\\
%   	        0 & \text{otherwise}
%	    \end{cases}
%    \]
%	    Then, $det(T) \not \equiv 0 \iff \text{G has a perfect matching}$.
%\end{theorem}


