\Lecture{Jayalal Sarma M.N.}{Aug 21, 2013}{3}{Algebraic Approach to
finding Perfect Matchings in Graphs}{Student}
\noindent
Preamble

\section{Application to Graph algorithms}
Consider the following problem of finding perfect matching.
\begin{definition}[Finding Perfect Matching]
Given a bipartite graph $G(V_1,V_2, E)$, we need to come up with an $E'
\subseteq E$ such that $\forall u \in V_1 \cup V_2$, there is exactly one edge
incident to it in $E'$.
\end{definition}

We shall give a polynomial formulation for the problem. Given $G(V_1, V_2, E)$
with vertex sets $|V_1| = |V_2| = n$. Define an $n\times n$ matrix $A$ where
$A(i,j) = 1$ if $(i,j) \in E$ and is $0$ otherwise for all $(i,j) \in V_1 \times
V_2$. Recall the determinant of $A$ given
by
\[ det(A) = \sum_{\sigma \in S_n} sign(\sigma) \prod_{i=1}^n A_{i,\sigma(i)}
\]

where \[
sign(\sigma) = \begin{cases} 
	  -1  & \text{ if } inv(\sigma) \text{ is even} \\
	  1 & \text{ if } inv(\sigma) \text{ is odd}
\end{cases}\]
and $inv(\sigma)$ is defined as 
$\left |\{(i,j) ~|~ i < j \text{ and } \sigma(i) > \sigma(j), 1 \le i<j \le n\}
\right |$\footnote{Every permutation can be decomposed to disjoint cycles. It
can also be shown that $sign(\sigma) = (-1)^{(n - \text{\# of odd cycles})}$}.

We denote $f(x) \equiv 0$ to denote that polynomial $f(x)$ is the zero
polynomial.
\begin{lemma}
For the matrix $A$ as defined as before, $det(A) \not \equiv 0 \implies
\text{$G$ has a perfect matching}$ \label{lem:det_pm_simple}
\end{lemma}
\begin{proof}
Let $det(A) \not \equiv 0$. Hence there exists a $\sigma \in
S_n$ such that $\prod_{i=1}^n A_{i, \sigma(i)} \ne 0$. Hence the edge set
$E' = \{(i, \sigma(i)) ~|~ 1\le i\le n\}$ exists in $G$ and since $\sigma(i) =
\sigma(j)$ iff $i=j$ for every $i,j$, $E'$ form a perfect matching.

\end{proof}
Note that converse of this statement is not true. For example, consider the
bipartite graph whose $A$ matrix is $A = 
\begin{bmatrix}
		1 & 1 & 0 & 0 \\
		0 & 1 & 1 & 0 \\
		0 & 0 & 1 & 1 \\
	        1 & 0 & 0 & 1
\end{bmatrix}
$. It can be verified that $det(A) = 0$ but the bipartite graph associated has
a perfect matching.

Hence the next natural question to ask is what kind of modification to the
converse of previous lemma (lemma~\ref{lem:det_pm_simple}) is true. In the
example considered, there were two perfect matchings in $G$ having opposite 
sign due to which the determinant became $0$. So the modification should
ensure that perfect matchings of opposite signs does not cancel off in the
determinant.

One way to achieve this is as follows.  Define a matrix $T$ as
\[ T(i,j) = \begin{cases}
		x_{ij} & (i, j) \in E \\
		      0 & \text{ otherwise}
	\end{cases}
\]
where $(i,j) \in V_1 \times V_2$.
Now, if we consider determinant of this matrix, we can see that the monomials
corresponding a $\sigma \in S_n$ can be a product of $n$ variables at most.
Hence $det(T)$ is a polynomial in $n^2$ variables with degree at most $n$.
We now give polynomial formulation for the problem of checking perfect
matching in a bipartite graph.
\begin{claim}
For the matrix $T$ as defined before, $det(T) \not \equiv 0 \iff \text{$G$
has a perfect matching}$
\end{claim}
\begin{proof}
($\Rightarrow$) Since $det(T) \not \equiv 0$, there exists a $\sigma \in S_n$
such that the monomial corresponding is non-zero and by definition of
deteminant it must be expressed as product of some $n$ set of variables
$\prod_i T(i,\sigma(i))$. The variable indices gives a matching and since
there are $n$ variables this is a perfect matching.

($\Leftarrow$) Suppose $G$ has a perfect matching given by a $\sigma$ and it
appears in $det(T)$ as $m$. Only way the term associated with this perfect
matching can vanishes from $det(T)$ is via another perfect matching which
appear as $-m$ in $det(T)$

But note that for any two different perfect matchings in $G$ if we look at the
permutations associated, they must differ in at least one edge. Hence the
associated terms must differ in some variable. Hence there does not exists
another matching which can vanish $m$ from $det(T)$ showing $det(T) \not \equiv
0$.
\end{proof}

Hence to check if the bipartite graph $G$ has a perfect matching or not, it
suffices to check if the polynomial $det(A)$ is identically zero or not.
Checking if a polynomial is identically zero or not is one of the fundamental
question in this area.

Note that this problem becomes easy if the polynomial is given as sum of 
monomial form. In most of the cases, the polynomial will not be given this way.
For example if we consider our problem, we are just given the $T$ matrix and
$det(T)$ is the required polynomial. Trying to expand $det(T)$ and simplifying
will involve dealing with $n!$ monomials which is not feasible.

So far we have spoken about finding perfect matching in bipartite graphs. A
similar result holds for general graphs also. 
\begin{theorem}
Given a graph $G(V,E)$ and for $i, j \in V$, $i \ne j$, define an $n \times n$
matrix $T$ as
\[T_{i,j} = \begin{cases}
		x_{ij} & (i,j) \in E, i < j\\
	       -x_{ji} & (i,j) \in E, i > j\\
   	        0 & \text{otherwise}
	    \end{cases}
    \]
	    Then, $det(T) \not \equiv 0 \iff \text{G has a perfect matching}$.
\end{theorem}


