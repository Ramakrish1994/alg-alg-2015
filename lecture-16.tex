\Lecture{Jayalal Sarma M.N.}{Sep 10, 2013}{16}{More on Fields}   

\noindent

Quick introduction to vector spaces. Viewing field extensions as vector spaces. Characterestic of a field. Sizes of fields. Constructing extensions and uniqueness of fields of a given size (up to isomorphism).

\section{Field Extensions as Vector Spaces}

A vector space over a field $\F$ is a set $V$ with two kinds of operations - addition and scalar multiplications - satisfying the following properties. Elements of $V$ are called vectors and elements of $\F$ are called scalars.
\begin{itemize}
\item $(V,+)$ forms an abelian group.
\item If $\alpha$ is a scalar, and $v$ is a vector, then $\alpha v$ is a vector.
\item If $\alpha$ is a scalar, and $u$ and $v$ are vectors, then $\alpha(u+v)$ is the same vector as $\alpha u+ \alpha v$.
\item If $\alpha_1$, $\alpha_2$ are scalars, and $v$ is a vector, then $\alpha_1(\alpha_2 v)$ is the same vector as $(\alpha_1 \alpha_2)v$ where $\alpha_1\alpha_2$ is the multiplication in $\F$.
\end{itemize}

Some easy examples are the set of points in $\R \times \R$. Set of polynomials of degree $d$ over a field $\F$ forms a vector space with the natural notion of addition and multiplication.

Let $E$ be a field and $F$ be a subfield of it. One can view $E$ as a vector space over $F$. To see this, view the elements of $F$ as scalars and the elements of $E$ as vectors in the above definition.

\subsection{Linear Independence, Basis, and Dimension}

\begin{definition}
A set of vectors $v_1, v_2, \ldots v_k$ are said to be \textit{linearly independent}, if:
\[ \alpha_1v_1 + \alpha_2 v_2 + \ldots + \alpha_k v_k = 0 \Longrightarrow \alpha_1 = 0 \land \alpha_2 = 0 \land \ldots \land \alpha_k = 0 \]
\end{definition}
For any set $S$ of vectors, the set of vectors spanned by it, denoted by  {\sc Span}$(S)$ is the set of vectors that can be expressed as the linear combination of vectors in $S$. A set $S$ is said to be a basis of a vector space, if $S$ itself is linearly independent and the {\sc Span}$(S)$ is the whole space.

We need two observations about the basis of a vector spaces and basis. 

All basis of a vector space are of the same size. Suppose there is an $S$ and an $S'$ which forms the basis of the same vector space $V$, and $|S| > |S'|$. Since $S$ and $S'$ are subsets of $V$ itself, the elements of $S'$ \jsay{This needs to be completed.}

Since all basis of a vector space are of the same size - it must be the case that.
\subsection{Minimal Polynomials - viewing adjoining as a vector space}

\section{Characterestic of Rings \& Fields}

Consider the following Ring homomorphism from $\Z$ to a ring $R$.
\[ \phi : \Z \to R \]
where, $\forall a \in \Z$, $\phi(a) = |a|.1$ if $a > 0$, and $\phi(a) = |a|.(-1)$ if $a < 0$. where $n.1$ is simply a notation for adding the identity of the ring $R$, $n$ times to itself.

We will first check that it is a ring homomorphism. \jsay{Yet to be written}

Let $I$ be the kernel of this map. Since $\Z$ is a principal ideal domain, $I$ is singly genereated and the generator is simply the least number in absolute value. Let $\ell$ be the generator of the ideal. We know that the ideal $I$ is simply $\ell \Z$. This $\ell$ is called the characterestic of the ring $R$. In other words, {\em characterestic of a ring $R$ with identity is simply the smallest number of times one needs to add $1$ to get to 0}. Indeed, it is possible that adding the identity to itself never gets to $0$ of the ring. In this case $I = {0}$ and $\ell=0$ - we say that the characterstic of the ring is $0$.

We explore more properties that can be derived from this $\ell$. 
Let $R'$ be the image of this homomorphism. We know that $R'$ is a subring of $R$. Consider the quotient ring $\Z/\ell\Z$ which is same as $\Z_\ell$. By the first isomorphism theorem we have the following:
$\Z_\ell \cong R'$. In other words, if the characterestic of a ring $R$ is $\ell$, then there is an isomorphic copy of $\Z_\ell$ sitting inside $R$ as a subring.

Now we turn to characterestic of a field. First of all let us argue that it can only be zero or a prime number.

\begin{lemma}
The characterestic of a field is either a prime number or zero.
\end{lemma}
\begin{proof}
Let $F$ be a field. Suppose the characterestic is not a prime and is $\ell \in \Z$. Assume for the constradiction that $\ell$ is not prime. $\ell = p.q$ where $p, q < n$. Indeed, $\ell$ is least integer such that $\ell.n = 0$. Hence $p.1 \ne 0$ and $q.1 \ne 0$. Since $\phi$ is a homomorphism, associated with the $\ell$ that we discussed above, $\phi(pq) = \phi(p)\phi(q)$. Since the LHS is $0$, $\phi(p)$ and $\phi(q)$ forms zero divisors in $\F$. Thus, we have arrived at a contradiction and hence the lemma.
\end{proof}

\begin{corollary}
Any finite field must have a subfield whose order is a prime number.
\end{corollary}
\begin{proof}
Let $F$ be a finite field. We first argue that the characterestic cannot be zero. If it is zero, then we know that the ideal $I$ in the above discussion is the zero ideal and hence the quotient ring is $\Z$ itself.
Hence, $\exists R' \subseteq F$ such that $Z \cong R'$, which implies that $\F$ must have infinite cardinality. Thus, characterestic can only be a prime number. Thus, an isomorphic copy of $Z_p$ for some prime $p$ must be present in every field.
\end{proof}

\section{Sizes of Finite Fields}

We combine the ideas developed in the previous two sections to conclude that the sizes of finite fields cannot be arbitrary.

\begin{lemma}
The size of any finite field is of always of the form $p^d$ for some prime $p$ and a non-negative integer $d$.
\end{lemma}
\begin{proof}
$\Z_p$ (for some prime $p$) appears a subfield(up to isomorphism) of any finite field. Let $d$ be the dimension of $F$ as a vector space over $\Z_p$. That is, there is a subset $S \subseteq F$ with $|S| = d$, which forms the basis of $F$ over $\Z_p$. Let us say that $S = \{ a_1, a_2, \ldots a_d \}$. Indeed, each vector (each element of $a \in \F$ can be viewed as a $d$-tuple $(\alpha_1, \alpha_2, \ldots, \alpha_d)$ such that $a = \alpha_1a_1 + \alpha_2a_2 + \ldots + \alpha_d a_d$. Can two tuples represent the same $a$? No, because it would mean then that $\sum_i \alpha_i a_i = \sum_i \alpha' a_i$. This contradicts the fact that $S$ is linearly independent (since it forms a basis of $\F$). Hence there are precisely $p^d$ tuples possible, each of them representing distinct elements of $\F$ as a vector space over $\Z_p$. Hence the size of the field $\F$ must be exactly $p^d$.
\end{proof}

\section{Constructing Field Extensions}

For any $p$ and $d$, is there a field of size $p^d$. We will answer this question positively.

Consider a polynomial $p(x)$ of degree $d$ that is irreducible over $\Z_p$. Let $a$ be a root of such a polynomial. Clearly $a \notin \Z_p$. Consider the field $\Z_p/\langle p \rangle$. This is isomorphic to $\Z_p(a)$ which also is a vector space over $\Z_p$. 

We argue that the size of this finite field is precisely $p^d$. First of all, we argue that any element of the space $\Z_p(a)$ can be viewed as a linear combination of elements in $\{1, a, a^2, \ldots, a^{d-1}\}$. We do not need an $a^d$ in this expression - indeed, it can be written as a combination of the other elements of lesser power since $p(a) = 0$. Suppose there is a linear combination of $(1, a, a^2, \ldots, a^{d-1})$ that goes to zero, that is $a$ is a root of the polynomial of lesser degree than $p(a)$. But then there is a polynomial of degree less than $p(x)$ which has $a$ as the root.

\section{Uniqueness of Fields up to isomorphism}

We will be greedy, for any $p$ and $d$, are there two non-isomorphisc fields of size $p^d$? We will answer this question negatively. So, we can always talk about {\em the} field of size $p^d$.
\jsay{Define splitting field etc.}
\begin{lemma}
The splitting field of a polynomial are always isomorphic to each other.
\end{lemma}
\begin{proof}
\jsay{Yet to be written}
\end{proof}

\begin{lemma}
For any field $\F$, there is a polynomial whose splitting field is $\F$.
\end{lemma}
\begin{proof}
Let $|\F| = k$. Consider the multiplicative group $\F - \{0\}$. Let $g$ be an element in  this group. We know by Lagrange's theorem, $g^{k-1} = 1$ where $1$ is the multiplicative identity. Thus for all $g \in \F$, $g^k = g$. Thus all of them are roots of the polynomial $x^k - x$, as a polynomial in $\F[x]$. Since this polynomial can have at most $k$ roots, the polynomial completely splits in $\F$ and it does not split in any subfield of $\F$. Hence $\F$ is the splitting field of the polynomial $x^k - x$. 
\end{proof}

By combining the above two lemmas, we get the main point of this section. That finite fields of a fixed size must be isomorphic.