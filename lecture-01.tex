\Lecture{Jayalal Sarma M.N.}{July 31, 2013}{1}{Introduction, Motivation and the Language}{Student}

\noindent
General Comments.

\section{Overview of the course. Administrative, Academic policies}


\section{Introduction and Motivation}
Main theme of this course is to use algebra to solve computational problems.
Let us consider the following two problems :
\begin{description}
	\item [Plagarism check]
	Given two $C$ programs $P_1$ and $P_2$ check if they are the same
	under renaming of variables
	\item [Molecule detection]
	Given two chemical moleculesm check if they have the same structure.
\end{description}

Note that both these problems can be modelled using a graph. For example, in
the second case one could view the molecule being given as adjacency matrix.
Our aim in both cases are similar which is to check if there is isomorphism
between two graphs.

\begin{definition}[Graph Isomorphism problem]
	Two graphs $G_1(V_1,E_1)$, $G_2(V_2,E_2)$ are said to be isomorphic if
	there is a bijective map $\sigma:V_1 \to V_2$ such that $\forall
	(u,v) \in V_1 \times V_1$, \[(u,v) \in E_1 \iff (\sigma(u), \sigma(v))
	\in E_2 \]
\end{definition}

One can also consider the following special case of the 

\section{Overview of the course}


