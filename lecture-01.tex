\Lecture{Jayalal Sarma M.N.}{July 31, 2013}{1}{Introduction, Motivation and
the Language}{K Dinesh}

\section{Overview of the course. Administrative, Academic policies}
\newpage

\section{Introduction and Motivation}
Main theme of this course is to use algebra to solve computational problems.
Let us consider the following two problems :
\begin{description}
	\item [Plagarism check]
	Given two $C$ programs $P_1$ and $P_2$ check if they are the same
	under renaming of variables
	\item [Molecule detection]
	Given two chemical moleculesm check if they have the same structure.
\end{description}

Note that both these problems can be modelled using a graph. For example, in
the second case one could view the molecule being given as adjacency matrix.
Our aim in both cases are similar which is to check if there is isomorphism
between two graphs.

\begin{definition}[Graph Isomorphism]
	Two graphs $X_1(V_1,E_1)$, $X_2(V_2,E_2)$ are said to be isomorphic if
	there is a bijective map $\sigma:V_1 \to V_2$ such that $\forall
	(u,v) \in V_1 \times V_1$, \[(u,v) \in E_1 \iff (\sigma(u), \sigma(v))
	\in E_2 \]
\end{definition}
\begin{problem}
	The \emph{graph isomorphism problem} is the decision problem of
	checking if given two graphs $X_1, X_2$ are isomorphic.
\end{problem}

We are also interested in the following special case of the above problem
called graph automorphism problem. 

\begin{definition}[Graph Automorphism]
	For a graph $X(V,E)$, an automorphism of $X$ is a renaming of the
	vertices of $X$ given by a bijective map $\sigma:V \to V$ such 
	that $\forall
	(u,v) \in V \times V$, \[(u,v) \in E \iff (\sigma(u), \sigma(v))
	\in E \]
\end{definition}
We are interested in the set of all bijections such that they are
automorphisms of $X$. We denote this by $Aut(X)$.
\begin{definition}
	For a graph $X$ on $n$ vertices, 
	$Aut(X) = \{ \sigma ~|~ \sigma :[n] \to [n], \sigma \text{ is an
	automorphism of }X \}$
\end{definition}

Note that an identity map which takes a vertex to itself always belongs to
$Aut(X)$ for all graphs $X$. Hence the question is are there any bijections
other than the identity map as automorphism of $X$.

\begin{problem}[Graph Automorphism Problem]
	Given a graph $X$ does $Aut(X)$ has any element other than the
	identity element.
\end{problem}

One way to see bijections is via permutations. This is because, every
bijection define a permutation and vice versa.

Let $X$ be an $n$ vertex graph. 
Denote $S_n$ to be the set of all permutations on $n$ elements. Hence $Aut(X)$
can be defined as $\left\{ \sigma ~|~ \sigma \in S_n \text{ and $\sigma$ is an
automorphism of } G\right\}$. 

We now show that the set $Aut(X)$ has some nice properties. Given $\sigma_1,
\sigma_2 \in Aut(X)$, we can compose two permutations as follows :$\sigma_1
\circ \sigma_2 = (\sigma_1(\sigma_2(1)),\sigma_1(\sigma_2(2)), \ldots,  
\sigma_1(\sigma_2(n)))$. This is same as applying $\sigma_2$ on identity
permutation and then applying $\sigma_1$ to the result. We show that
$Aut(X)$ along with the composition operation $\circ$
gives us many nice properties.
\begin{itemize}
	\item If $\sigma_1, \sigma_2 \in Aut(X)$, then $\sigma_1 \circ
		\sigma_2$ is also an automorphism of $X$. The reason is that
		for any $(u,v) \in X \times X$, $(u,v) \in E \iff
		(\sigma_2(u), \sigma_2(v)) \in E$ Now applying $\sigma_1$ on
		the previous tuple, we get that $(\sigma_2(u), \sigma_2(v)) 
		\iff (\sigma_1 \circ \sigma_2(u), \sigma_1 \circ \sigma_2(v))
		\in E$. Hence $(u,v) \in E \iff (\sigma_1 \circ \sigma_2(u),
		\sigma_1 \circ \sigma_2(v)) \in E$.
		This tells that $Aut(X)$ is \emph{closed} under $\circ$.
	\item The composition operation is also \emph{associative}.
	\item \emph{Identity} permutation belongs to $Aut(X)$ as observed
	before.  
	\item Since we are considering bijections, it is natural to
		consider \emph{inverse} for a permutation $\sigma$ denoted
		$\sigma^{-1}$ with the property that $\sigma \circ
		\sigma^{-1}$ is identity permutation. 

\end{itemize}
We gave a definition of inverse for an arbitrary permutation. But then a
natural question is : given $\sigma \in Aut(X)$, is it true that $\sigma^{-1}$
also belongs to $Aut(X)$. It turns out that it is true.
\begin{claim}
	For the graph $X(V,E)$
	$\sigma \in Aut(X) \iff \sigma^{-1} \in Aut(X)$
\end{claim}
\begin{proof}
	Recall that $\sigma \in Aut(G)$ iff $\forall (u,v) \in V \times V$,
	$(u,v) \in E \iff (\sigma(u), \sigma(v)) \in E$. In particular, this
	must be true for $(\sigma^{-1}(u), \sigma^{-1}(v))$ also. This means,
	\[ (\sigma^{-1}(u), \sigma^{-1}(v)) \in E \iff (\sigma(\sigma^{-1}(u)),
	\sigma(\sigma^{-1}(v))) \in E \]
	By definition of $\sigma^{-1}$ we get that 
	$(\sigma^{-1}(u), \sigma^{-1}(v)) \in E \iff (u,v) \in E$. This shows
	that $\sigma^{-1} \in Aut(X)$ 
\end{proof}
Objects which satisfy these kind of properties are called groups.

\section{Overview of the course}
There are two major themes. 
\begin{itemize}
	\item Algorithms for permutation groups.
	\item Algorithms for polynomials.
\end{itemize}
