\Lecture{Jayalal Sarma}{Aug 4, 2015}{2}{Algebraic Approach to
Primality Testing}{K Dinesh}
In this lecture, we will see an algebraic approach to solving a fundamental
problem in Number Theory.
\section{Application to Number Theory}
Following is an algorithmic question that we are interested.
\begin{problem}
Given a number $n$ in its binary representation, check if it is a prime or not
in time $O(poly(\log N))$.
\end{problem}
\begin{note}
The trivial algorithms that we can think of will depend on $n$ and hence takes
time exponential in its input representation.
\end{note}

Consider the following property about prime number proved by Fermat which is
of interest in this context.
\begin{theorem}[Fermat's Little Theorem] If $N$ is a prime, then $\forall~a, 1
	\le a \le N-1$, \[ a^N = a \mod N \]
\end{theorem}
\begin{proof}
	Fix an $a \in \{1,2,\ldots N-1\}$. Now consider the sequence $a, 2a,
	\ldots, (N-1)a$. The question we ask is : can any two of the numbers
	in this sequence be the same modulo $N$. We claim that this cannot
	happen. We give a proof by contradiction : suppose that there are two
	distinct $r,s$ with $1 \le r < s \le N-1$ and $sa = ra \mod N$. Then
	clearly $N | (s-r)a$ which means $N | a$ or $N | (s-r)$. But both
	cannot happen as $a, s-r$ are strictly smaller than $N$.

	This gives that all the $N$ numbers in our list modulo $N$ are
	distinct. Hence all numbers from $1,2,\ldots, N-1$ appear in the list
	when we go modulo $N$. Taking product of the list and the list modulo
	$N$, we get 
	\[ (N-1)! a^{N-1} = (N-1)! \mod N \]
	By cancelling $(N-1)!$, we get that $a^{N-1} = 1 \mod N$.
\end{proof}
This tells that the above condition is necessary for a number to be prime. If
this test is also sufficient (i.e, if the converse of the above theorem is
true), we have a test for checking primality of a number. But it turns out
that this is not true due to the existance of Carmichael numbers which are not
prime numbers but satisfy the above test.

So one want a necessary and sufficient condition which can be used for
primality testing. For this, we need the notion of polynomials.
\begin{definition}
	A polynomial $p(x) = \sum_{i=0}^d a_ix^i$ with $a_d \ne 0$ denotes a
	polynomial in one variable $x$ of degree $d$. Here $a_i$s are called
	coefficients and each term excluding the coefficient is called a
	monomial. \textbf{A polynomial is said to be identically zero, if all
	the coefficients are zero.}
\end{definition}

A polynomial time algorithm for this problem has been found in 2002 by
Manindra Agarwal, Neeraj Kayal and Nitin Saxena (called as the AKS algorithm). 
In their result, they used the following polynomial characterisation for a
prime number.
\begin{theorem}[Polynomial formulation (Agarwal-Biswas 1999)] Let $N \ge 1$ be
	an integer. Define a polynomial 
	$p_N(z) = (1+z)^N - 1 - z^N$. Then \[ p_N(z) \equiv 0 (mod~ N)
	\iff N \text{ is prime} \]
\end{theorem}

Hence checking if $N$ is prime or not boils down to checking if $p_N(z)$ is
identically $0$ or not except for the fact that the underlying operations are
done modulo $N$.

Proof of the theorem is as follows.
\begin{proof}
	Note that $p_N(z) = \sum_{i=1}^{N-1} \binom{N}{i} z^i$ and
	$\binom{N}{i} = \frac{N(N-1)\ldots (N-i+1)}{1\cdot 2 \ldots i}$ with $1
	\le i \le N-1$. 

	If $N$ is prime, then $\binom{N}{i} = N \times k_i$ for some integer
	$k_i$ as none of $1, 2, \ldots i$ divides $N$. Hence $\binom{N}{i}
	\mod N = 0$ for every $i$ and $p_N(z) \equiv 0 \mod N$ since the
	polynomial has all coefficients as zero.

	If $N$ is composite, we need to show that $p_N(z) \not 
	\equiv 0 (mod~ N)$ which means that there is at least one non-zero
	coefficient $p_N(z) \mod N$\footnote{In class we asked the following
		question of finding an $a \in \{1,2,\ldots, N-1\}$ such that
	$p_N(a) \ne 0 \mod N$. This has a counter example. Consider the case
	where $N$ is a Carmicahel number. By definition, Carmichael numbers
	are composite numbers that satisfy Fermat's Litte theorem.  Hence if
	$N$ is Carmichael,  $\forall~ a \in \{1,2,\ldots N-1 \}$, we get  $a^N
	= a \mod N$. This gives that $\forall~ a \in \{1,2,\ldots, N-1\}$
	\[	p_N(a)  = ((a+1)^N-a^N -1) \mod N = ((a+1)-a-1) \mod N \]
	which is zero modulo $N$.}. 	
	Since $N$ is composite, there exists a prime $p$ such that $p
	~|~ N$.  Let $p^k | N$ where $k \ge 1$ is the largest exponent of $p$
	in the prime factorisation of $n$. Hence $p^{k+1} \nmid N$.
	
	We first show that $p_N(z)$ is a non zero polynomial by showing that
	the coefficient of $z^i$ for $i = p$, which is $\binom{N}{p}$, is non
	zero modulo $p^k$ showing\footnote{Note that if
		$N|\binom{N}{i}$ then $p^k | \binom{N}{i}$. Taking
		contrapositive, we get that $\binom{N}{i} \not = 0 \mod p^k$
		implies $\binom{N}{i} \not = 0 \mod N$} $p_N(z)$ is not a zero
		polynomial modulo $N$. 
	Let $N = g \times p^k$. In $\binom{N}{p}$, $p$ of the denominator 
	divides $N$. Note that $p$ cannot divide $g$, for if it does, then
	$p^{k+1} | N$ which is not possible by choice of $p$ and $k$.
		\begin{equation}
		\binom{N}{p} = \frac{g\times p^k(N-1)\ldots (N-p+1)}
		{1\cdot 2 \ldots p} = \frac{g\times p^{k-1}(N-1)\ldots 
			\label{eq:ag-bis-proof}
		(N-p+1)}{1\cdot 2 \ldots p-1} \end{equation}
	Hence it must be that $p^{k-1}$ divides $\binom{N}{p}$. This is
	because, there is no $p$ left now in the denominator to divide $N$.
	Now $p^k \nmid \binom{N}{p}$ because, none of the terms $(N-1),
	\ldots,
	(N-p+1)$ can have $p$ as a factor since all these terms are obtained by
	subtracting at most $p-1$ times from $N$. Hence $\binom{N}{p} \ne 0
	\mod p^k$. This completes the proof.
\end{proof}

Note that Fermat's Little Theorem is a special case of Agarwal-Biswas
formulation of primality testing.
\begin{claim}
Fermat's Little Theorem is a special case of Agarwal-Biswas theorem.
\end{claim}
\begin{proof}
	To prove Fermat's Little theorem, we need one direction of implication
	of Agarwal-Biswas theorem : 
	\begin{center}
	``If $N$ is prime, then $(1+z)^N = (1+z^N) \mod N$''.
	\end{center}

	Note that Fermat's Little theorem asks about $a^N \mod N$ for $a \in
	\{1,2,\ldots,N-1\}$. Since the implication talks about $z^N \mod N$
	and $(1+z)^N \mod N$, this suggests an induction strategy on $a$.

	Let $N$ be prime. We check the base case : for $a=1$, the
	$1^N =1 \mod N$. Hence the base case is true. By induction, suppose
	that $a^N = a \mod N$ for $a \in \{1,2,\ldots, N-1\}$. Now, 
	\begin{align*}
		(a+1)^N & = (a^N +1) \mod N && [\text{By Agarwal-Biswas as $N$
		is prime}]\\
		& = (a+1) \mod N && [\text{By inducive hypothesis}]
	\end{align*}
	This completes the inductive case. 
\end{proof}

This shows that checking if the polynomial $p_N(z)$ is a zero polynomial is an
if and only if check for primality of $N$. There are two naive ways to do it.
One is to evaluate it at all the numbers from $1$ to $N$ or to expand out the
polynomial and see if it is identically zero. But both operations are costly.

So checking primality of a number now boils down to checking if a polynomial
is identically zero or not. This is a fundamental problem of polynomial
identity testing. We will be discussing about polynomial identity testing and 
AKS algorithm in our next theme.
