\Lecture{Jayalal Sarma M.N.}{Aug 2, 2013}{2}{Algebraic Approach to
Primality Testing}{Student}
In this lecture, we will see an algebraic approach to solving a fundamental
problem in Number Theory.
\section{Application to Number Theory}
Following is an algorithmic question that we are interested.
\begin{problem}
Given a number $n$ in its binary representation, check if it is a prime or not
in time $O(poly(\log n))$.
\end{problem}
\begin{note}
The trivial algorithms that we can think of will depend on $n$ and hence takes
time exponential in its input representation.
\end{note}

Consider the following property about prime number proved by Fermat which is
of interest in this context.
\begin{theorem}[Fermat's Little Theorem] If $N$ is a prime, then $\forall~a, 1
	\le a \le N-1$, \[ a^N = a \mod N \]
\end{theorem}
\begin{proof}
	Fix an $a \in \{1,2,\ldots N-1\}$. Now consider the sequence $a, 2a,
	\ldots, (N-1)a$. The question we ask is : can any two of the numbers
	in this sequence be the same modulo $N$. We claim that this cannot
	happen. We give a proof by contradiction : suppose that there are two
	distinct $r,s$ with $1 \le r < s \le N-1$ and $sa = ra \mod N$. Then
	clearly $N | (s-r)a$ which means $N | a$ or $N | (s-r)$. But both
	cannot happen as $a, s-r$ are strictly smaller than $N$.

	This gives that all the $N$ numbers in our list modulo $N$ are
	distinct. Hence all numbers from $1,2,\ldots, N-1$ appear in the list
	when we go modulo $N$. Taking product of the list and the list modulo
	$N$, we get 
	\[ (N-1)! a^{N-1} = (N-1)! \mod N \]
	By cancelling $(N-1)!$, we get that $a^{N-1} = 1 \mod N$.
\end{proof}
This tells that the above condition is necessary for a number to be prime. If
this test is also sufficient (i.e, if the converse of the above theorem is
true), we have a test for checking primality of a number. But it turns out
that this is not true due to the existance of Carmichael numbers which are not
prime numbers but satisfy the above test.

So one want a necessary and sufficient condition which can be used for
primality testing. For this, we need the notion of polynomials.
\begin{definition}
	A polynomial $p(x) = \sum_{i=0}^d a_ix^i$ with $a_d \ne 0$ denotes a
	polynomial in one variable $x$ of degree $d$. Here $a_i$s are called
	coefficients and each term excluding the coefficient is called a
	monomial.
\end{definition}

An algorithm for this problem has been found in 2002 by Manindra Agarwal,
Neeraj Kayal and Nitin Saxena. In their result, they used the following
polynomial characterisation for a prime number.
\begin{theorem}[Polynomial formulation (Agarwal-Biswas 1999)] Let $n \ge 1$ be
	an integer. Define a polynomial 
	$p_n(z) = (1+z)^n - 1 - z^n$. Then \[ p_n(z) \equiv 0 (mod~ n)
	\iff n \text{ is prime} \]
\end{theorem}
Hence checking if $n$ is prime or not boils down to checking if $p_n(z)$ is
identically $0$ or not except for the fact that the underlying operations are
done modulo $n$.

Proof of the theorem is as follows.
\begin{proof}
	Note that $p_n(z) = \sum_{i=1}^{n-1} \binom{n}{i} z^i$ and
	$\binom{n}{i} = \frac{n(n-1)\ldots (n-i+1)}{1\cdot 2 \ldots i}$ with $2
	\le i \le n-1$. 
	If $n$ is prime, then $\binom{n}{i} = n \times k_i$ for some integer
	$k_i$ as none of $1, 2, \ldots i$ divides $n$. Hence $\binom{n}{i}
	\mod n = 0$ for every $i$ and $p_n(z) \equiv 0 \mod n$.

	If $n$ is composite, then there exists a prime $p$ such that $p ~|~
	n$.  Let $p^a \nmid n$ where $a \ge 1$ is the largest exponent of $p$ in
	the prime factorisation of $n$. Hence $p^{a+1} \nmid n$.
	
	To show that $p_n(z)$ is a non zero polynomial in this case, we 
	show that the coefficient of $z^i$ for $i = p$, which is
	$\binom{n}{p}$, is non zero modulo $p^a$ showing\footnote{Note that if
		$n|\binom{n}{i}$ then $p^a | \binom{n}{i}$. Taking
		contrapositive, we get that $\binom{n}{i} \not = 0 \mod p^a$
		implies $\binom{n}{i} \not = 0 \mod n$} $p_n(z) \not \equiv 0
		\mod n$.
	

	\begin{observation}
	Let $k$ be an integer with $n = k \times p^a$. Note that $p \nmid k$
	for if it does then $p^{a+1} | n$. \label{obs:ag-bis-proof-fact}
	\end{observation}
	We need the following claim.
	\begin{claim}
		$p$ does not divide $(n-1)(n-2)\ldots (n-p+1)$. 
		\label{cl:ag-bis-main}
	\end{claim}
	\begin{proof}
		\begin{equation}
		\binom{n}{p} = \frac{k\times p^a(n-1)\ldots (n-p+1)}
		{1\cdot 2 \ldots p} = \frac{k\times p^{a-1}(n-1)\ldots 
			\label{eq:ag-bis-proof}
		(n-p+1)}{1\cdot 2 \ldots p-1} \end{equation}


		Suppose we assume that there is an $i$ such that $p | n-i$.
		Then $p|i$ (as $p|n$). But since $1 \le i \le p-1$ it is not
		possible that $p|i$. Hence for every $i$, $p \nmid n-i$.
	\end{proof}

	If $p^a$ divides $\binom{n}{p}$, then it must be that from
	equation~\ref{eq:ag-bis-proof}, $p$ divides $\frac{k(n-1) \ldots
	(n-p+1)}{1\cdot 2\ldots (p-1)}$. Hence $p | k$ or $p | (n-1)
	\ldots (n-p+1)$. But we have already shown that both are not possible
	(observation~\ref{obs:ag-bis-proof-fact} and
	claim~\ref{cl:ag-bis-main}). Hence $p^a \nmid \binom{n}{p}$.

\end{proof}

